\documentclass[times
%,specification
%,annotation
%,titlepage
]{itmo-student-thesis}

%% Опции пакета:
%% - specification - если есть, генерируется задание, иначе не генерируется
%% - annotation - если есть, генерируется аннотация, иначе не генерируется
%% - times - делает все шрифтом Times New Roman, собирается с помощью xelatex
%% - languages={...} - устанавливает перечень используемых языков. По умолчанию это {english,russian}.
%%                     Последний из языков определяет текст основного документа.

%% Делает запятую в формулах более интеллектуальной, например:
%% $1,5x$ будет читаться как полтора икса, а не один запятая пять иксов.
%% Однако если написать $1, 5x$, то все будет как прежде.
%\usepackage{icomma}

%% Один из пакетов, позволяющий делать таблицы на всю ширину текста.
\usepackage{tabularx}

\usepackage{svg}
\usepackage{tikz}
\usetikzlibrary{arrows}
%\usepackage{filecontents}
\usepackage{caption}
\usepackage{subcaption}
\usepackage{bibentry}

\def\MyResearchTitle{Поддержка вызова функций и передачи объектов между реализациями языков со статическими и динамическими типизациями в среде выполнения на платформе Ark}

\def\MyResearchAim{Разработка межъязыкового взаимодействия на платформе Ark между реализациями языков со статическими и динамическими типизациями с минимизацией неуправляемого кода}

\def\MyResearchTargets{
	\MyResearchTargetNext{обзор существующих подходов в области межъязыкового взаимодействия}{;}
	\MyResearchTargetNext{выделение затрагиваемых компонент платформы и их архитектурных ограничений}{;}
	\MyResearchTargetNext{проектирование интерфейса межъязыкового взаимодействия}{;}
	\MyResearchTargetNext{реализация прототипа, минимизирующего количество неуправляемого кода}{;}
	\MyResearchTargetNext{оценка результатов с точки зрения корректности и производительности}{.}
}

\lstdefinelanguage{etspseudo}{
	keywords={let, const, typeof, new, true, false, catch, function, return, null, catch, switch, var, if, in, while, do, else, case, break},
	keywordstyle=\color{blue}\bfseries,
	ndkeywords={class, export, boolean, throw, implements, import, this, void, int, long, short, double},
	ndkeywordstyle=\color{purple}\bfseries,
	identifierstyle=\color{black},
	sensitive=false,
	comment=[l]{//},
	morecomment=[s]{/*}{*/},
	commentstyle=\color{purple}\ttfamily,
	stringstyle=\color{red}\ttfamily,
	morestring=[b]',
	morestring=[b]"
}

\def\MyResearchTargetNext#1#2{\item #1#2}

\setcounter{tocdepth}{2}
\def\Indy{\texttt{invokedynamic}}

\begin{filecontents}[overwrite]{bachelor-thesis.bib}
%@article{ example-russian,
%	author      = {Максим },
%	title       = {Генерация тестов},
%	journal     = {Научно-технический вестник {СПбГУ} {ИТМО}},
%	number      = {2(72)},
%	year        = {2011},
%	pages       = {72-77},
%	langid      = {russian}
%}
%
%@book{ bellman,
%	author      = {R. E. Bellman},
%	title       = {Dynamic Programming},
%	address     = {Princeton, NJ},
%	publisher   = {Princeton University Press},
%	numpages    = {342},
%	pagetotal   = {342},
%	year        = {1957},
%	%langid      = {english}
%}

@inproceedings{ one-vm-to-rule-them-all,
	doi = {10.1145/2509578.2509581},
	url = {https://doi.org/10.1145/2509578.2509581},
	year = {2013},
	month = oct,
	publisher = {{ACM}},
	author = {Thomas W\"{u}rthinger and Christian Wimmer and Andreas W\"{o}{\ss} and Lukas Stadler and Gilles Duboscq and Christian Humer and Gregor Richards and Doug Simon and Mario Wolczko},
	title = {One {VM} to rule them all},
	booktitle = {Proceedings of the 2013 {ACM} international symposium on New ideas,  new paradigms,  and reflections on programming and software}
}

@online{ what-i-learned-from-lua-jit,
	year        = {2016},
	title       = {What I learned from LuaJIT},
	author      = {Vyacheslav Egorov},
	url         = {https://mrale.ph/talks/vmss16/#/},
	%langid      = {english},
	urldate     = {2023-05-11}
}

@online{ ignition-iterpreter,
	year        = {2018},
	title       = {Firing up the Ignition interpreter},
	author      = {{v8 developers}},
	url         = {https://v8.dev/blog/ignition-interpreter},
	%langid      = {english},
	urldate     = {2023-05-11}
}

@online{ luajit,
	year        = {2023},
	title       = {LuaJIT FFI library},
	author      = {{luajit developers}},
	url         = {https://luajit.org/ext_ffi.html},
	%langid      = {english},
	urldate     = {2023-05-11}
}

@online{ golang-stack,
	year        = {2014},
	title       = {How Stacks are Handled in Go},
	author      = {Daniel Morsing},
	url         = {https://blog.cloudflare.com/how-stacks-are-handled-in-go/},
	%langid      = {english},
	urldate     = {2023-05-11}
}

@online{ v8-pointer-tagging,
	year        = {2020},
	title       = {Pointer Compression in V8},
	author      = {Igor Sheludko, Santiago Aboy Solanes},
	url         = {https://v8.dev/blog/pointer-compression},
	%langid      = {english},
	urldate     = {2023-05-11}
}

@online{ luajit-nan-box,
	year        = {2022},
	title       = {LuaJIT VM tags, values and objects},
	author      = {Mike Pall},
	url         = {https://github.com/LuaJIT/LuaJIT/blob/224129a8e64bfa219d35cd03055bf03952f167f6/src/lj_obj.h#L240},
	%langid      = {english},
	urldate     = {2023-05-11}
}

@online{ torque,
	year        = {2023},
	title       = {V8 Torque user manual},
	author      = {{v8 contributors}},
	url         = {https://v8.dev/docs/torque},
	%langid      = {english},
	urldate     = {2023-05-11}
}

@online{ erlang-gc,
	year        = {2023},
	title       = {Erlang Garbage Collector},
	author      = {{Ericsson AB}},
	url         = {https://www.erlang.org/doc/apps/erts/garbagecollection},
	%langid      = {english},
	urldate     = {2023-05-11}
}

@online{ zig-cinclude,
	year        = {2023},
	title       = {Zig Language Reference},
	author      = {Andrew Kelley},
	url         = {https://ziglang.org/documentation/0.10.1/#cInclude},
	%langid      = {english},
	urldate     = {2023-05-11}
}

@online{ project-panama-overview,
	year        = {2020},
	title       = {Project Panama: как сделать Java «ближе к железу»?},
	author      = {Leader-ID},
	url         = {https://habr.com/ru/companies/leader-id/articles/505072/},
	langid      = {russian},
	urldate     = {2023-05-11}
}

@online{ java-lang-void,
	year        = {2014},
	title       = {Class Void},
	author      = {{Oracle Corporation}},
	url         = {https://docs.oracle.com/javase/8/docs/api/java/lang/Void.html},
	%langid      = {english},
	urldate     = {2023-05-11}
}

@online{ electron-isolates,
	year        = {2023},
	title       = {Context Isolation},
	author      = {{OpenJS Foundation and Electron contributors}},
	url         = {https://www.electronjs.org/docs/latest/tutorial/context-isolation},
	%langid      = {english},
	urldate     = {2023-05-11}
}

@online{ jvm-invokedynamic,
	year        = {2017},
	title       = {Understanding Java method invocation with invokedynamic},
	author      = {Ben Evans},
	url         = {https://blogs.oracle.com/javamagazine/post/understanding-java-method-invocation-with-invokedynamic},
	%langid      = {english},
	urldate     = {2023-05-13}
}

@online{ overload-typescript,
	year        = {2023},
	title       = {More on Functions},
	author      = {{Typescript contributors}},
	url         = {https://www.typescriptlang.org/docs/handbook/2/functions.html},
	%langid      = {english},
	urldate     = {2023-05-14}
}

@online{ overload-python,
	year        = {2021},
	title       = {The Correct Way to Overload Functions in Python},
	author      = {Martin Heinz},
	url         = {https://martinheinz.dev/blog/50},
	%langid      = {english},
	urldate     = {2023-05-14}
}

@online{ overload-elixir,
	year        = {2023},
	title       = {Elixir. Getting started. Modules and functions},
	author      = {{The Elixir Team}},
	url         = {https://elixir-lang.org/getting-started/modules-and-functions.html},
	%langid      = {english},
	urldate     = {2023-05-14}
}

@online{ jni-obj-references,
	year        = {2023},
	title       = {Overview of JNI object references},
	author      = {{IBM Corporation}},
	url         = {https://www.ibm.com/docs/en/sdk-java-technology/8?topic=collector-overview-jni-object-references},
	%langid      = {english},
	urldate     = {2023-05-14}
}

@online{ beam-proc,
	year        = {2020},
	title       = {Elixir on Erlang VM demystified},
	author      = {Kamil Lelonek},
	url         = {https://blog.lelonek.me/elixir-on-erlang-vm-demystified-320557d09e1f},
	%langid      = {english},
	urldate     = {2023-05-14}
}

@online{ graal-langs,
	year        = {2023},
	title       = {GrallVM Language Implementations},
	author      = {{Oracle Corporation}},
	url         = {https://github.com/oracle/graal/blob/master/truffle/docs/Languages.md},
	%langid      = {english},
	urldate     = {2023-05-23}
}

@online{ j2v8,
	year        = {2021},
	title       = {j2v8 V8 bindings},
	author      = {{EclipseSource}},
	url         = {https://github.com/eclipsesource/J2V8/blob/00dddaa31a80782abbe93c4a01f325db3c4975d6/src/main/java/com/eclipsesource/v8/V8.java#L1579},
	%langid      = {english},
	urldate     = {2023-05-23}
}

@book{java8-spec,
  address = {Upper Saddle River, NJ},
  author = {Gosling, James and Joy, Bill and Steele, Guy L. and Bracha, Gilad and Buckley, Alex},
  biburl = {https://www.bibsonomy.org/bibtex/2e93895f98a08e3e26fcfc2e2d4673370/flint63},
  description = {1. Auflage 1996},
  edition = 5,
  file = {Oracle eBook:2014/GoslingJoyEtAl14.pdf:PDF;InformIT Product page:http\://www.informit.com/title/013390069X:URL;Amazon Search inside:http\://www.amazon.de/gp/reader/013390069X/:URL;Safari:https\://www.safaribooksonline.com/library/view/the-java-virtual/9780133922745/:URL},
  groups = {public},
  isbn = {978-0-13-390069-9},
  keywords = {01841 103 book shelf safari software java},
  publisher = {Addison-Wesley},
  series = {Java Series},
  title = {The Java Language Specification: Java SE 8 Edition},
  url = {http://docs.oracle.com/javase/specs/},
  username = {flint63},
  year = 2014
}

@techreport{cowlishaw2008standard,
  added-at = {2020-10-06T14:35:39.000+0200},
  address = {New York, NY, USA},
  biburl = {https://www.bibsonomy.org/bibtex/217408cd727ef3e502b41d9d0e8502e60/jaeschke},
  editor = {Cowlishaw, Mike},
  institution = {IEEE Computer Society},
  interhash = {7e3b85491aa7f46d2cc79d2ece76304a},
  intrahash = {17408cd727ef3e502b41d9d0e8502e60},
  keywords = {arithmetic ddm floating ieee mk3.3 point standard},
  month = aug,
  number = {IEEE Std 754-2008},
  timestamp = {2021-05-05T09:10:37.000+0200},
  title = {IEEE Standard for Floating-Point Arithmetic},
  type = {Standard},
  url = {https://web.archive.org/web/20160806053349/http://www.csee.umbc.edu/~tsimo1/CMSC455/IEEE-754-2008.pdf},
  year = 2008,
	urldate = {2023-05-12}
}

@inproceedings{graalvm-polyglot,
  doi = {10.1145/2816707.2816714},
  url = {https://doi.org/10.1145/2816707.2816714},
  year = {2015},
  month = oct,
  publisher = {{ACM}},
  author = {Matthias Grimmer and Chris Seaton and Roland Schatz and Thomas W\"{u}rthinger and Hanspeter M\"{o}ssenb\"{o}ck},
  title = {High-performance cross-language interoperability in a multi-language runtime},
  booktitle = {Proceedings of the 11th Symposium on Dynamic Languages},
	urldate = {2023-05-12}
}

@inproceedings{streamfusion,
	author = {Coutts, Duncan and Leshchinskiy, Roman and Stewart, Don},
	year = {2007},
	month = {09},
	pages = {315-326},
	title = {Stream Fusion. From Lists to Streams to Nothing at All},
	volume = {42},
	journal = {Sigplan Notices - SIGPLAN}
}

\end{filecontents}
\addbibresource{bachelor-thesis.bib}

\lstdefinelanguage{none}{
  identifierstyle=
}

\lstset{
	showtabs=false,
	breaklines=true,
	language=etspseudo,
}

\begin{document}

\tikzstyle{every picture}+=[remember picture]
\tikzstyle{na}=[shape=rectangle,inner sep=0pt,text depth=0pt]

\newcommand{\TODO}{{ \color{red} TODO }}

\def\MyResize#1#2{\resizebox{\ifdim\width>#1#1\else\width\fi}{!}{#2}}

%\renewcommand\includesvg[2][]{AAAA}

\studygroup{M34361}
\title{\MyResearchTitle}
\author{Прокопенко Кирилл Дмитриевич}{Прокопенко К.Д.}
\supervisor{Беляев Евгений Александрович}{Беляев Е.А.}{кандидат технических наук}{факультет ИТиП, доцент}
\publishyear{2023}
%% Дата выдачи задания. Можно не указывать, тогда надо будет заполнить от руки.
\startdate{29}{декабря}{2022}
%% Срок сдачи студентом работы. Можно не указывать, тогда надо будет заполнить от руки.
\finishdate{31}{мая}{2023}
%% Дата защиты. Можно не указывать, тогда надо будет заполнить от руки.
\defencedate{15}{июня}{2023}

\addconsultant{Соломенников Д.И.}{без сетепени, без звания}

\secretary{Павлова О.Н.}

%% Задание
%%% Техническое задание и исходные данные к работе
\technicalspec{Требуется добавить на платформу Ark возможность запускать динамически типизированные программы из статически типизированного языка, вызывать функции и методы над статически типизированными объектами из динамического языка и наоборот, передавать объекты в обе стороны}

%%% Содержание выпускной квалификационной работы (перечень подлежащих разработке вопросов)
\plannedcontents{Пояснительная записка должна содержать исследование существующих решений в области межъязыкового взаимодействия и предлагать новое составное решение, подходящее для платформы Ark. В нем должны поддерживаться вызовы методов и передача объектов наравне с их конструированием. Взаимодействие между виртуальными машинами должно происходить с минимальным использованием нативного кода или слоев-посредников. Требуется валидация и проверка результатов как с точки зрения корректности, так и с точки зрения производительности}

%%% Исходные материалы и пособия
\plannedsources{\begin{enumerate}
	\item \fullcite{one-vm-to-rule-them-all};
	\item \fullcite{what-i-learned-from-lua-jit};
	\item внутренняя документация компании по работе с платформой Ark.
\end{enumerate}}

%%% Цель исследования
\researchaim{\MyResearchAim.}

%%% Задачи, решаемые в ВКР
\researchtargets{\begin{enumerate}
	\MyResearchTargets
\end{enumerate}}

%%% Использование современных пакетов компьютерных программ и технологий
\addadvancedsoftware{Набор текстов \texttt{Sunspider} для тестирования производительности}{Раздел~\ref{sec:sunspider}}
%\addadvancedsoftware{Пакет \texttt{biblatex} и программное средство \texttt{biber}}{Список использованных источников}

%%% Краткая характеристика полученных результатов
\researchsummary{Было предложено составное решение, описывающее как изменения конкретных компонент платформы, позволяющие работать нескольким виртуальным машинам с общим набором объектов, так и общее устройство межъязыкового взаимодействия. Был реализован работоспособный прототип, демонстрирующий предложенный подход и позволяющий сократить время выполнения программы из набора тестов производительности \texttt{Sunspider} более чем в два раза.}

%%% Гранты, полученные при выполнении работы
\researchfunding{Данная работа выполнялась во время работы в некоторой компании. Гранты получены не были.}

%%% Наличие публикаций и выступлений на конференциях по теме выпускной работы
\researchpublications{По теме работы в данный момент ничего не опубликовано. Однако, статья была представлена на XII Конгрессе Молодых Ученых в ИТМО и подана для публикации в сборник трудов Конгресса.
\begin{refsection}
%\nocite{example-english, example-russian}
\printannobibliography
\end{refsection}
}

\maketitle{Бакалавр}
\tableofcontents

\startprefacepage

Данная работа рассматривает взаимодействие языков программирования во время исполнения~--- возможность взаимодействия программного кода и структур данных, описанных на нескольких языках программирования внутри одной исполняемой программы.

Межъязыковое взаимодействие является важной технологией по нескольким причинам:

\begin{enumerate}
\item переиспользование кода: межъязыковое взаимодействие позволяет использовать ранее написанный код на другом языке программирования. Например, оно может сделать доступной графическую библиотеку, написанную на статически типизированном языке для вызова из динамически типизированного, без требования написания дополнительного кода или генерации промежуточных файлов. Подобное может быть проделано и в обратную сторону: уже существующая библиотека может быть переписана на более стабильном (обнаруживающем большее число ошибок во время компиляции) или производительном (дающим больший контроль над исполнением) языке. Переиспользование кода может быть полезно и для постепенной миграции кодовой базы на другой язык программирования;

\item гибкость решений: в то время, как многие языки программирования позиционируются как <<языки общего назначения>>, это не значит, что они не подходят для некоторых задач лучше, чем для других. Например, языки, удовлетворяющие открытому стандарту ecmascript, популярны в сфере разработки графических интерфейсов не только из-за того, что они были единственной альтернативой в интернет-браузерах на протяжении многих лет, но и благодаря упрощениям, которые дает динамическая типизация и наследование прототипов. Все перечисленное ведет к тому, что компаниям легче нанять специалиста на таком языке, нежели искать специалиста на других. Таким образом, возможность выбирать язык, подходящий для задачи, позволяет сделать решения более гибкими;

\item сотрудничество команд: когда несколько команд или разработчиков работают над одним проектом, им может быть удобно использовать разные языки программирования. Возможность их взаимодействия обеспечивает работу в одной программе или приложении, что повышает эффективность разработки.
\end{enumerate}

Перечисленные пункты для компаний влекут финансовую прибыль, так как ускоряют разработку и упрощают найм сотрудников, т.е. снижают стоимость производства программного обеспечения.

В то время как межъязыковое взаимодействие не является полностью решенной задачей, количество проектов, ставящих своей целью его упрощение, указывает на спрос в данной сфере.

Таким образом, целью данной работы является \MakeLowercase\MyResearchAim.

Поставленная цель может быть разбита на следующие задачи:
\begin{enumerate}
	\MyResearchTargets
\end{enumerate}

Настоящая работа организована следующим образом: в Главе~\ref{ch:overview} приводится обзор предметной области, анализируются существующие решения и выделяются их недостатки, приводится отличия платформы Ark. В Главах \ref{ch:platform} и \ref{ch:architecture} добавление межъязыкового взаимодействия на платформу Ark решается за счет разделения задачи на две практически независимые части:
\begin{enumerate}
	\item внесение <<низкоуровневых>> изменений в платформу, позволяющих виртуальным машинам работать с объектами и методами принадлежащими другим виртуальным машинам;
	\item разработку набора протоколов, по которым виртуальные машины <<договариваются>> о том, как происходит взаимодействие (что включает в себя, например, конверсии типов).
\end{enumerate}

В Главе~\ref{ch:results} приводятся методы оценки результатов, включающие замеры производительности на общепринятом тесте и сравнение с аналогами.

\chapter{Обзор существующих реализаций межъязыкового взаимодействия}\label{ch:overview}

\startrelatedwork

\section{Термины и понятия}
В данном разделе представлены термины, используемые в других частях представленной работы. В дальнейшем приведенные понятия будут использоваться в указанных значениях, если не указано обратное.
\def\MakeTerm#1#2{\textit{#1}~--- #2.\par}
\subsection{Специфичные для области термины}
	\MakeTerm{Управляемый код (managed code)}{код, исполняемый под управлением виртуальной машины. Однако, код среды исполнения так же не будет считаться управляемым в данной работе, поскольку к нему нет доступа у сборщика мусора}
	\MakeTerm{Управляемый язык}{язык, исполнение которого происходит под управлением виртуальной машины}
	\MakeTerm{NaN упаковка (nan boxing)}{подход к представлению данных, позволяющий избежать аллокаций. Более подробное описание представлено в секции~\ref{nan-boxing-explanation}}
	\MakeTerm{Деоптимизация (deoptimization)}{процесс возвращения из скомпилированного кода в интерпретируемый в случае, когда какое-то предположение оказалось неверным. При этом скомпилированный код может быть инвалидирован}
	\MakeTerm{Встраиваемые кэши (inline caches)}{оптимизация интерпретаторов динамических языков, позволяющая ускорять выполнение некоторых инструкций за счет кэширования прошлых результатов}
	\MakeTerm{Интринсик (intrinsic)}{встроенная в компилятор специальная команда}
	\MakeTerm{Сохраненное состояние (save state)}{место в программе, где известно какие регистры и какие адреса на стеке соответствуют какому состоянию интерпретатора: регистрам, стеку и номеру инструкции интерпретатора. Это состояние может потребоваться, например, для деоптимизации}
	\MakeTerm{Безопасная точка (safe point)}{место в программе, где она может остановиться в частности для некоторого этапа сборки мусора. Это означает, что это так же и сохраненное состояние}
	\MakeTerm{Мета-трассировка (meta-tracing)}{подход к JIT компиляции, при котором создается интерпретатор абстрактного синтаксического дерева (потенциально, со специальными командами, о которых знает JIT компилятор), который затем компилируется в предположении, что код неизменен}
	\MakeTerm{Сериализация (serialization)}{процесс конвертации структур в памяти в последовательность байт}
	\MakeTerm{Десериализация (deserialization)}{процесс обратный сериализации}
	\MakeTerm{Сборка мусора (garbage collection)}{процесс поиска в программе недостижимых выделенных фрагментов памяти и их очистка. Недостижимые объекты в данном случае называются <<мусором>>, а код, выполняющий данный процесс <<сборщиком мусора>>}
	\MakeTerm{Синтаксический сахар (syntactic sugar)}{синтаксические возможности, упрощающие написание программ, но не привносящие в язык программирования новые возможности}
	\MakeTerm{Встраиваемые кэши (inline caches)}{техника оптимизации, при которой инструкции байткода снабжаются модифицируемыми ячейками памяти, которые интерпретатор может использовать для укорения исполнения за счет кэширования метаданных, полученных при прошлых выполнениях инструкции}
	\MakeTerm{Коверканье имен (name mangling)}{техника, при которой в имя функции дополнительно кодируется метаинформация, такая как типы аргументов}
	\MakeTerm{Исходный уровень компиляции (англ. baseline)}{компиляция байткода в нативный один в один, без применения оптимизаций, зачастую сводится к вызову интринсика}
	\MakeTerm{Фаза остановки мира (stop-the-world)}{фаза сборки мусора, при которой основное приложение и все его потоки полностью остановлены}

\subsection{Аббревиатуры}
	\MakeTerm{ABI (Application Binary Interface)}{бинарный интерфейс приложения: соглашение по которому машинный код интерпретирует данные}
	\MakeTerm{JIT компиляция (Just in Time compilation)}{подход, при котором программа компилируется во время исполнения}
	\MakeTerm{AOT компиляция (Ahead of Time compilation)}{подход, при котором программа компилируется один раз перед исполнением}
	\MakeTerm{JVM (Java Virtual Machine)}{конкретная спецификация виртуальной машины}
	\MakeTerm{v8}{конкретная виртуальная машина реализующая стандарт ecmascript}
	\MakeTerm{FFI (Foreign Function Interface, интерфейс внешней функции)}{интерфейс позволяющий вызывать функции из динамической библиотеки написанные на другом языке}
	\MakeTerm{\texttt{int32}, \texttt{float32} и так далее}{обозначения типов для чисел число в конце означает число бит; знаковых целые числа обозначаются как \texttt{int}; числа с плавающей точкой (стандарт IEEE-754~\cite{cowlishaw2008standard})~--- как \texttt{float}}

\subsection{Дополнительные понятия}
	\MakeTerm{Гонка данных (data race)}{ситуация, при которой несколько потоков исполнения обращаются к общей памяти без надлежищих синхронизаций, что ведет к недетерминизму и ошибкам}
	\MakeTerm{Ручное управление памятью}{подход к управлению памятью, когда каждый выделенный участок памяти должен быть удален ровно один раз и при помощи кода, написанного разработчиком}
	\MakeTerm{\Indy}{инструкция байткода JVM~\cite{jvm-invokedynamic}. Данная инструкция совершает вызов метода и параметризуется в частности прототипом, однако, в отличие от остальных инструкций вызова, она не имеет конкретного метода, известного при компиляции, а вместо этого при первом исполнении вызывает некоторую функцию под названием <<BootstrapMethod>>, которая создает <<CallSite>>, т.е. описание точки вызова. CallSite могут быть нескольких видов, например, если результирующая точка вызова является константной, то вся инструкция \Indy может быть JIT скомпилирована в код с производительностью равной обычному вызову. Такой подход позволяет более оптимально компилировать вызовы динамических методов или соединять статически типизированный язык с динамически типизированными конструкциями за счет параметризации сигнатурой}
	\MakeTerm{False sharing}{ситуация, при которой два или более потоков обращаются по разным адресам памяти, которые, однако, расположены на одной кэш-линии, что приводит к ухудшению производительности}

\section{Существующие реализации межъязыкового взаимодействия}

Все существующие реализации можно разбить на три категории по признаку управляемости взаимодействующих языков, которые будут рассмотрены ниже.

\subsection{Реализации взаимодействия двух неуправляемых (нативных) языков}
Взаимодействие языков в данной категории зачастую сводится к поддержке совместимости с ABI языка C, и поскольку данная задача является сложной из-за отсутствия данных об интерфейсе в самих бинарных файлах, известные решения зачастую включают целые языки: C++ и Zig, которые, по сути, позволяют включать целые заголовочные файлы самого языка C. В случае C++ для функций необходима пометка \texttt{extern "C"} для того чтобы отключить коверканье имен глобальных символов, а в языке Zig достаточно указать интринсик \texttt{@cInclude}, который разберет заголовочный файл при помощи библиотеки Clang и автоматически создаст в памяти декларации, понимаемые языком Zig~\cite{zig-cinclude} (однако, включение одного и того же файла два раза подряд может привести к несовместимости с точки зрения типовой системы).

\subsection{Реализации взаимодействия управляемых языков с неуправляемыми}
\subsubsection{JNI --- Java Native Interface}
Чтобы справиться с перемещающим сборщиком мусора, все указатели на объекты Java кучи помещаются в специальное хранилище, о котором знает сборщик мусора~\cite{jni-obj-references}. В результате, нативный код взаимодействует только с косвенными указателями. Из-за того, что нативный код не гарантирует расстановку безопасных точек с какой-либо регулярностью, на время всего исполнения нативного кода поток погружается в безопасное состояние, а все функции, обладающие доступом к Java куче, замедляются из-за того, что должны возвращаться из этого безопасного состояния. Для ускорения программ существуют некоторые <<критические>> функции, такие как \texttt{GetPrimitiveArrayCritical}, которые дают прямой указатель, а не косвенный (т.е. к которому есть доступ без обращения к JNI), но могут заблокировать все Java приложение до тех пор, пока этот указатель не будет освобожден.

Доступа из Java к нативной куче\footnote{Java куча~--- место, где выделяются Java объекты; нативная~--- где выделяются объекты нативным кодом} нет, что обязывает создавать нативные функции доступа, которые, в свою очередь, работают медленнее, чем <<аналогичные>> написанные на Java из-за невозможности встраивания и погружения в безопасное состояние. К другим ограничениям, для нативных функций необходимо генерировать специальные заголовочные файлы.

\subsubsection{Project panama}
Проект старается бороться с ограничениями JNI~\cite{project-panama-overview}: добавляет доступ из Java к объектам на нативной куче (условно по указателям, но проще и безопаснее, чем \texttt{sun.misc.Unsafe}); добавляет поддержку нативных функций с переменным числом аргументов (в Java они бы запаковались в массив, в то время как в нативном соглашении о вызовах передались бы в небезопасной манере: через регистры или стек, без контроля за числом аргументов, что, конечно, менее безопасно, однако имеет свои применения).

Декларация структур с нативным расположением полей может помочь и обычным Java приложениям, так как виртуальная машина имеет право переупорядочивать поля, а задание строгого расположения может помочь избежать, например, false sharing'а за счет добавления неиспользуемых полей.
\subsubsection{Haskell FFI}
Интерфейс внешних фнукций в языке программирования Haskell позволяет вызывать функции языка C с примитивами и указателями, что делает его крайне ограниченным и однонаправленным. Однако, есть возможность доступа к полям структур через смещение и размер, что, в сочетании с аппликативом, позволяет относительно удобно реализовывать сериализацию и десереализацию в структуры языка Haskell. Большинство реализаций FFI имеют схожий ограниченный функционал.
\subsubsection{LuaJIT FFI}\label{intro:luajitffi}
LuaJIT FFI выделяется среди интерфейсов внешних функций, поскольку добавляет полноценный синтаксический анализатор деклараций языка C~\cite{luajit}, благодаря чему расширяет типовую систему lua добавлением нового типа \texttt{cdata} в добавок к таблицам, числам и так далее. Данный подход позволяет вызывать нативные функции и обращаться к полям нативных структур с той же производительностью, что и изначальный язык --- C (за исключением проверок типов) благодаря эффективной JIT компиляции и простоте языка lua.
\subsubsection{CPPYY}
Библиотека CPPYY позволяет вызывать различные конструкции языка C++ из Python при помощи использования проекта cling: интерпретатора и JIT компилятора C++, построенного на основе LLVM. В то время, как он является удобной заменой автоматическим генератором <<клея>> (файлов, описывающих доступные функции), и может быть использован для выноса крупных частей приложения на C++ для ускорения, он не устраняет стоимость вызова процедур в Python, и может ее даже увеличивать за счет необходимости преобразований типов и связанных с этим выделений памяти (например, \texttt{int} в Python это большое число, но в рамках перехода между языками оно может быть преобразовано к \texttt{int} из C, что, наряду с проверкой типов, будет требовать запаковку или распаковку). Замер скорости на программе в приложении~\ref{apx:cppyy-bench} демонстрирует этот эффект.

\subsection{Реализации взаимодействия управляемых языков}
Данная работа принадлежит именно этому разделу, и он является наиболее интересным для реализации, поскольку межъязыковое взаимодействие значительно упрощается, если обе реализации языков находятся под управлением одной виртуальной машины, поскольку это позволяет избежать многих проблем, таких как наличие двух сборщиков мусора.
\subsubsection{Виртуальные машины JVM и .NET}
Многие языки (такие как Kotlin) имеют относительно близкую к <<главному>> языку (Java или C\#) систему типов, что позволяет компилировать их в байткод общей виртуальной машины (JVM) без значительной потери производительности. Другие же языки должны вносить компромиссы в свой дизайн, как, например, \texttt{null} значения в языке F\#, чтобы быть полностью совместимыми с другими языками платформы .NET и позволить постепенную миграцию на допускающий меньшее число ошибок во время исполнения язык.
\subsubsection{Nashorn}
Nashorn это реализация стандарта ecmascript на JVM, которая использует исключения для создания деоптимизаций и инструкцию \Indy для поддержки динамической типизации. В то время, как на простых мономорфных тестах данный проект может догнать v8, на более сложных полиморфных примерах начинают становиться видны недостатки <<позитивных типов>> и нехватка специализированного динамического профиля.

\subsubsection{GraalVM}
GraalVM это новейшая виртуальная машина, построенная на основе HotSpot (реализация JVM), полагающаяся на мета-трассирующие оптимизации. Несмотря на хорошие результаты с точки зрения простоты разработки нового языка (на что указывает количество разработанных реализаций для популярных языков~\cite{graal-langs}), реализация которого будет обладать высокой пиковой производительностью, данный подход, к сожалению, имеет некоторые недостатки. В частности, реальным проектам важны и другие характеристики, такие как объем потребляемой памяти, время загрузки, скорость интерпретации и энергопотребление на компиляцию. Наличие целого абстрактного дерева и его интерпретация может быть не позволительна, например, в браузерах на мобильных платформах. Это подробно обсуждается в презентации по добавлению интерпретатора ignition в v8~\cite{ignition-iterpreter}.

\section{Обобщенный наивный подход к взаимодействию виртуальных машин}
Проблему межъязыкового взаимодействия было бы возможно решить за счет создания объекта-посредника для каждого объекта другого языка, который бы вызывал нативный код среды выполнения, а она бы в свою очередь за счет библиотеки рефлексии выполняла нужное действие: вызывала метод или считывала поле.

Однако, такое решение обладает проблемами с управлением памятью, свойственными взаимодействию управляемых языков с неуправляемыми (поскольку взаимодействие происходит через нативный код), совершенно не эффективно (поскольку переключение контекста на нативный и использование рефлексии это достаточно дорогие операции в сравнении с простым чтением поля объекта) и может ломать, например, рефлексивность равенства: чтение поля ссылочного типа может каждый раз возвращать новый \texttt{Proxy} объект. Тем не менее, оно используется в проектах наподобие j2v8, реализующем взаимодействие Java и ecmascript (v8) при помощи JNI~\cite{j2v8}.

%Данный подход не использовал бы то, что оба языка находятся в общей управляемой среде и обладал бы недостатками, свойственными взаимодействию управляемого и неуправляемого кода.

\section{Недостатки существующих реализаций}
Когда один из языков не находится под управлением виртуальной машины, либо они находятся под управлением разных, возникает множество сложностей, связанных, например, со сборкой мусора, которая при перемещающем сборщике мусора ведет к накладным расходам на каждое обращение к объекту.

Если же рассматривать решения, использующие одну виртуальную машину, то зачастую эта виртуальная машина оптимизирована под один конкретный язык, что крайне ограничивает языко-специфичные подходы, такие как NaN упаковка, используемая в LuaJIT~\cite{luajit-nan-box} или пометка указателей, используемая в v8~\cite{v8-pointer-tagging}, поскольку они требуют низкоуровневой поддержки со стороны различных компонент виртуальной машины. По этой же причине расширение стека для корутин так же не представляется возможным, поскольку реализуется через расширенный пролог метода и модификацию сборщика мусора для миграции стеков, хотя используется в эталонных реализациях таких языков как Elixir и Golang \cite{golang-stack}.

\finishrelatedwork

\section{Что отличает платформу Ark?}\label{ch:ark-good}
Платформа Ark объединяет в себе несколько виртуальных машин, реализации которых различны и могут иметь разные, никак не связанные, среды исполнения, наборы инструкций и представления объектов, делающие их несовместимыми, однако эти реализации имеют общие компоненты, что оставляет возможность для добавления межъязыкового взаимодействия. Например, управление памятью и сборщик мусора почти полностью унифицированы. Такой подход позволяет реализовать каждую виртуальную машину в наиболее эффективной манере: с сохранением таких подходов как NaN запаковка; но не позволяет получить такое же простое взаимодействие языков на уровне байткода и устоявшейся системы типов как, например, JVM.

До начала данной работы межъязыковое взаимодействие на платформе полностью отсутствовало.

\chapterconclusion
В данной Главе был проведен обзор существующих решений в области межъязыкового взаимодействия, выявлены их недостатки, описано, как платформа Ark старается их решить, и чего ей не хватает для этого.

\chapter{Изменение компонент платформы Ark и детали их реализации}\label{ch:platform}
Как показано на рисунке~\ref{fig:all-components}, предложенная реализация межъязыкового взаимодействия затрагивает практически все компоненты платформы Ark, за исключением AOT компилятора, в котором нет информации профиля, необходимой для оптимизации межъязыкового взаимодействия. Процесс компиляции с одной стороны может быть отделен от исполнения, поскольку не взаимодействует напрямую с ее конструкциями, но с другой должен предоставлять среде исполнения данные, такие как расположение данных на стеке, и получать от нее информацию профиля в интерпретируемом для асинхронного компилятора виде.
\begin{figure}[!h]
	\caption{Компоненты платформы, участвующее в выполнении кода}\label{fig:all-components}
	\centering
	\resizebox{0.55\textwidth}{!}{\includesvg{build/res/dot/components.dot}}
\end{figure}

\section{Представление объектов на статически типизированной стороне}
Для представления объектов на статически типизированной стороне дерево типов было модифицировано как показано на рисунке~\ref{fig:statically-typed-tree}. Была добавлена новая ветвь иерархии классов, представляющая динамические объекты (\texttt{DynamicObject} на рисунке), которая не связана с корнем статических объектов (\texttt{Object}). Возможно добавление общего предка~--- \texttt{SomeObject}, который по техническим причинам нельзя расширять другими классами (является \texttt{sealed} в терминологии Java).

\begin{figure}[!h]
	\caption{Модификация иерархии статически типизированного языка}\label{fig:statically-typed-tree}
	\centering
	\includesvg{build/res/dot/statically-typed-tree.dot}
\end{figure}
Проверки типов и их преобразования также должны отличаться от обычных, что связано с особенностями реализации: в обеих реализациях тип представляющий класс объекта наследует некоторый \texttt{BaseClass}, содержащий, например, информацию для сборщика мусора. Однако, это не позволяет ни использовать библиотеку рефлексии, ни преобразовывать их друг к другу, поскольку последующее представление в памяти уже различно. Необходимо добавить команду компилятора, читающую указатель на \texttt{BaseClass} и берущую флаги из нее, где указано динамический ли класс у объекта. Так как \texttt{Object} и \texttt{DynamicObject} никак не связаны, компилятору следует требовать преобразование через \texttt{SomeObject}, как это реализовано, например, в F\#. Преобразование к \texttt{SomeObject} может быть соблюдено без каких либо проверок, поскольку этот класс является лишь синтетической абстракцией, без какого-либо представления во время выполнения.

\texttt{DynamicObject} может иметь некоторые расширения, такие как доступ к свойствам через точку или квадратные скобки, который будет компилироваться в нечто особенное, или иметь большее число наследников для выражение лучшей типовой безопасности.

\section{Представление объектов на динамически типизированной стороне}\label{nan-boxing-explanation}
В динамически типизированных языках любое значение может быть представлено как указатель на некоторый \texttt{ObjectHeader}, однако, такой подход потребляет значительное количество памяти из-за необходимости постоянных аллокаций и индирекций, что может еще и привести к удлинению пауз сборщика мусора. Из-за этого многие реализации динамически типизируемых языков~\cite{luajit-nan-box,v8-pointer-tagging} используют одну из двух стратегий: пометка указателей (pointer tagging) или NaN упаковка (NaN boxing). Платформа Ark использует второе. Эта техника опирается на то, что согласно стандарту IEEE-754~\cite{cowlishaw2008standard} число с плавающей точкой двойной точности представляется как экспонента из всех единиц, и единичный бит мантиссы тоже выставляется в единицу, чтобы сделать его не сигнализирующим. В предположении, что все NaN значения, которые может сгенерировать процессор имеют нулевую мантиссу (кроме бита на сигнализирующий ли, <<qNaN Indefinite>> на x86 процессоре), остаются не занятыми 50 бит, в которые можно поместить некоторый <<тег>> и <<значение>>. Эти значения представлены в таблице~\ref{fig:nan-explain}. При помощи сжатия указателя или специального аллокатора можно добиться того, чтобы любой указатель на управляемый объект помещался в оставшееся количество бит.\par
\begin{table}[!h]
	\caption{Число с плавающей точкой и NaN упаковка.}\label{fig:nan-explain}
	\centering
	\begin{tabular}{|c|c|c|c|c|c|}
		\hline
		число бит & 1 & 11 & \multicolumn{3}{c|}{52} \\
		\hline
		значение & знак & экспонента & \multicolumn{3}{c|}{мантисса} \\
		\hline
		NaN & знак & 1\dots 1 & \multicolumn{3}{c|}{мантисса} \\
		\hline
		qNaN indefinite & знак & 1\dots 1 & 1 & \multicolumn{2}{c|}{0\dots0} \\
		\hline
		кодирование & знак & 1\dots 1 & 1 & тэг & значение \\
		\hline
	\end{tabular}
\end{table}
Таким образом, для представления инородных объектов можно ввести новый тег, по которому будет храниться ссылка на статически типизированный объект. Для представления же статических сущностей можно использовать полиморфизм по типу объекта, т.е. сделать некоторую <<перегрузку>> этой семантики: ввести специальное поведение для случая, когда хранимая ссылка указывает на \texttt{Class} (объект, представляющий тип).

Данное изменение схоже с подходом, используемом в проекте LuaJIT FFI для представления структур языка C и описанном в Разделе~\ref{intro:luajitffi}, однако платформа Ark использует модификацию NaN упаковки, описанную в таблице~\ref{fig:arknanboxold}, и позволяющую хранить указатели на объекты без запаковки, что усложняет расширение, которое тем не менее возможно по схеме, указанной в таблице~\ref{fig:arknanboxnew}. Такое расширение сохраняет большую часть проверок, позволяет упаковывать объекты без модификации и проверять принадлежит ли объект куче унифицированным для обоих объектных тэгов способом.

\begin{table}
\centering
\caption{NaN упаковка на платформе Ark до модификации}\label{fig:arknanboxold}
\begin{tabular}{|c|c|c|}
\hline
число бит & 16 & 48 \\
\hline
значение & <<тэг>> & нагрузка \\
\hline
объект & \texttt{0000} & указатель \\
\hline
float64 & \begin{tabular}{c} \texttt{0001} \\ --- \\ \texttt{fffe} \end{tabular} & \dots \\
\hline
int32 & \texttt{ffff} & int32 \\
\hline
\end{tabular}
\end{table}

\begin{table}
\centering
\caption{NaN упаковка на платформе Ark после модификации}\label{fig:arknanboxnew}
\begin{tabular}{|c|c|c|c|}
\hline
число бит & 16 & \multicolumn{2}{c|}{48} \\
\hline
значение & <<тэг>> & \multicolumn{2}{c|}{нагрузка} \\
\hline
объект & \texttt{0000} & \texttt{0} & указатель \\
\hline
инородный объект & \texttt{0000} & \texttt{1} & указатель \\
\hline
float64 & \begin{tabular}{c} \texttt{0001} \\ --- \\ \texttt{fffe} \end{tabular} & \multicolumn{2}{c|}{\dots} \\
\hline
int32 & \texttt{ffff} & \multicolumn{2}{c|}{int32} \\
\hline
\end{tabular}
\end{table}

\section{Встраиваемые кэши}
Для ускорения выполнения некоторых инструкций, они имеют встраиваемые кэши, данные из которых затем используются оптимизирующим компилятором как профиль. Например, инструкция байткода \texttt{LdObjByName}, которая загружает свойство объекта по константному имени, сохраняет в себе динамический класс и число, представляющее смещение на искомое поле в нем. Таким образом, сначала кэш пустой, и во время исполнения необходимо узнать данные о классе: он представляется массивом или хэш-таблицей, затем при помощи подходящего алгоритма найти искомое значение ключа, обновить встраиваемый кэш и загрузить запрашиваемое значение. При повторном выполнении будет достаточно сравнить класс и сразу же загрузить значение в случае его совпадения, без совершения всех длительных действий и обращений к структурам данных, которые совершает медленный путь исполнения.

Встраиваемые кэши должны быть расширены для поддержки инородных объектов за счет добавления еще одного медленного пути: проверки, что объект инородный, затем обращения к другой виртуальной машине с целью получения обработчика, либо вызов всех накопленных в кэшах обработчиков до тех пор, пока какой-то из них не подойдет. Такой подход требует виртуального вызова на каждую обработку, однако позволяет унифицировать разные языки за одним интерфейсом. Код интерпретатора для, например, загрузки поля из статически типизированного объекта не может быть сокращен настолько же значительно, насколько и у динамически типизированного, поскольку необходима конвертация значения, зависящего от поля, в NaN упакованное, что в любом случае обязано привести к виртуальному вызову или условной проверке какую конверсию выбрать, что будет <<медленным>> путем.

Можно заметить, что и функции и типы, изначально хранимые во встраиваемых кэшах изначально были неподвижными (для сборщика мусора) и с бесконечным временем жизни, так как это необходимо для работы асинхронного JIT компилятора. Из-за чего и новые данные, хранимые там так же должны удовлетворять этим свойствам, что добавляет новый класс корневых объектов ко всей виртуальной машине.

\section{Работа с полями объектов}\label{ch:work-with-fields}
В то время, как на первый взгляд может показаться, что для загрузки значения поля объекта достаточно просто вставлять проверку на инородный тег, расширение известного класса, а затем запаковывать полученное значение в соответствующее примитивное (\texttt{int32}, \texttt{float64}, \texttt{bool}) или очередной объект (что, тем не менее, контролируется типовыми системами языков, описанными в Главе~\ref{ch:architecture}), несколько новых оптимизаций необходимы. Основная из них продемонстрирована на рисунке~\ref{fig:field-opt}: удаление распаковки после запаковки с несколькими нюансами

Во-первых, запакованное значение должно остаться в сохраненном состоянии, потому что занимает некоторый виртуальный регистр $v_i$, который может быть использован в случае дальнейшей деоптимизации. Однако, эта лишняя <<запаковка>> вероятно будет удалена при последующих оптимизациях, если они сочтут это безопасное состояние не используемым.

Во-вторых, так как инородный объект это ссылочный тип, его в не NaN упакованном виде тоже необходимо положить в последующее безопасное состояние, чтобы в случае сборки мусора ссылка на него обновилась. Так как ему не соответствует ни один виртуальный регистр, он помещается в так называемый мост (bridge).

И, наконец, последующие проверки на принадлежность к классу (\texttt{IsInstance}) могут быть статически изменены на проверки на \texttt{null}, поскольку JIT компилятор изначально полагается на компилятор статически типизированного языка, и в аналогичном коде двойного доступа к полю вставляет только проверки на нулевой указатель, но не на тип операнда. Эта проверка на нулевой указатель может быть и дальше удалена за счет информации о типах из статически типизированного языка, если он ее предоставляет, или заменена на обработку сигнала операционной системы.

Сохранение полей не является содержательным с точки зрения модификаций платформы и сводится к проверкам типов и сохранению поля в память.

\begin{figure}[!h]
	\caption{Оптимизация запаковки и распаковки инородных объектов.}\label{fig:field-opt}
	\centering
	\MyResize{\textwidth}{
			\begin{tabular}{ |c|c| }
				\hline
				До оптимизации & После оптимизации \\
				\hline
				\includesvg{build/res/dot/load-field-1.dot} & \includesvg{build/res/dot/load-field-2.dot} \\
				\hline
			\end{tabular}
		}
	\end{figure}

\section{Управление потоками} \label{sec:thread-management}
Для исполнения управляемого кода необходимы дополнительные данные (в пример можно привести буферы аллокации локальные для потока), которые инкапсулируются в то, что в данной работе будет называться <<состоянием потока>>. Это состояние во время интерпретации хранится в переменной локальной для потока, а в управляемом коде передается через регистр. Важно заметить, что при пересечении языковой границы необходимо сохранить его консистентность, что добавляет новое действие в некоторые места, такие как мост перехода из скомпилированного кода в интерпретатор, поскольку без использования межъязыкового взаимодействия значение состояния потока является константным после первого присвоения. Такой переход возможен, если каждый поток будет осведомлен о положении переменной локальной для потока, которой он принадлежит. При такой реализации не может возникнуть гонок данных, так как все действия происходят внутри одного реального потока.

Большинство динамически типизированных языков либо не позволяют создавать потоки (Lua, ecmascript), либо имеют глобальный лок интерпретатора (CPython, Ruby MRI). Есть несколько языков с поддержкой многопоточности, но они либо фокусируются на производительности (Julia), либо построены вокруг распределенности (Erlang, Elixir, в которых даже сборщик мусора работает внутри одного потока~\cite{erlang-gc}). Из этого было бы возможно вывести ограничение, что динамически типизированный язык должен взаимодействовать только с привязанным к этому же нативному потоку состоянием потока статически типизированного языка. На самом деле, такое ограничение является довольно распространенным и присутствует так же в библиотеках на подобие OpenGL.

Управление потоками важно для сборки мусора, поскольку существуют фазы остановки мира, в которые все потоки, выполняющие управляемый код, должны остановиться. Это может быть реализовано за счет, например, блокировки чтения-записи (readers-writer lock): исполняемые потоки берут право на чтение, а желающий их остановить на запись, и тогда его попадание в критическую секцию будет означать, что все потоки остановлены. Из-за того, что необходимо выполнять два состояния потока разных языков на одном реальном, принадлежащем операционной системе, может возникнуть путаница или неправильное количество держателей блокировки на чтение, и более удобной архитектурой было бы иметь отдельно потоки выполнения которые бы исполняли корутины, как это реализовано, например, в вирутальной машине Beam~\cite{beam-proc}.

\section{Вызов методов}
Как сказано в Разделе \ref{sec:thread-management}, у каждого языка есть свое <<состояние потока>>. Так как в скомпилированном коде указатель на это состояние лежит в регистре, и передается из функции в функцию по соглашению о вызовах, изменение могло бы быть закодировано как обычный \texttt{mov} известной константы до и после вызова, однако при обработке исключений и разворачивании стека данное состояние необходимо вернуть, чего можно добиться за счет создания дополнительного промежуточного стекового кадра. Его расположение внутри кадра метода, вызывающего инородную функцию, представлено на рисунке~\ref{fig:method-stack} и обозначено как <<Дополнительный кадр>>. Данный кадр может быть заполнен единожды в прологе метода, с последующим обновлением адреса возврата перед каждым вызовом. Для восстановления своего состояния потока после нормального возврата вызова инородного метода может быть использован этот же кадр.
\begin{figure}[!h]
\caption{Новое расположение данных в стековом кадре метода}\label{fig:method-stack}
\centering
\begin{tabular}{|c|@{}cr@{}|}
	\hline
	\multicolumn{2}{|c}{Адрес возврата}&\tikz\node[na](frameUpper){};\\
	\hline
	\multicolumn{2}{|c}{Указатель на предыдущий кадр}&\\
	\hline
	\multicolumn{2}{|c}{Слот метаданных}&\\
	\hline
	\multicolumn{2}{|c}{Локальные переменные}&\\
	\hline
	Дополнительный кадр&\multicolumn{2}{@{}c@{}|}{\begin{tabular}{c}
	Адрес возврата\\
	\hline
	Указатель на предыдущий кадр\tikz\node[na](frameLink){};\\
	\hline
	Специальное число\\
	\hline
	Указатель на состояние потока\\
	\end{tabular}}\\
	\hline
	\multicolumn{2}{|c}{Стековые аргументы}&\\
	\hline
\end{tabular}

\begin{tikzpicture}[remember picture,overlay,cyan,rounded corners,>=stealth,shorten > =1pt,shorten <=1pt,thick]
	\draw[->, color=black] (frameLink.east) to [bend right=45] (frameUpper.east);
\end{tikzpicture}

\end{figure}

\section{Сборка мусора}
\subsection{Пометка объектов}
Сборка мусора на платформе Ark по большей степени обобщена, однако самые низкоуровневые примитивы, которые, например, размечают и обходят объекты обязаны быть разными. Они параметризованы конфигурацией языка, которая, помимо прочего, указывает динамический ли язык. Для максимального переиспользования кода, данная структура может быть применено нетривиальное (и потенциально виртуальное) наследование, как показано на рисунке~\ref{fig:gc-markers} (зеленым выделены переименованные фрагменты).

Изначально код, размечающий объекты, имел \texttt{assert}, что класс полученного объекта соответствует контексту, и эта проверка удалялась в итоговой сборке платформы, поскольку включала атомарное чтение и условие в коде, который может выполняться десятки тысяч раз. Таким образом, чтобы не совершать ненужных действий, была введена новая конфигурация (уже не языка, а платформы), которая позволяет смешать маркеры (объекты проводящие пометку) для статически типизированного языка и динамически типизированного. Данная конфигурация используется в случае, когда разрешено межъязыковое взаимодействие, и указывает маркерам зависящим от типизации вставлять проверку на принадлежность объекта и переключать маркер на другой режим в случае несовпадения. При выключенном межъязыковом взаимодействии может все так же срабатывать \texttt{assert}, который будет удален.

Можно выдвинуть гипотезу, что переключение контекста происходит крайне редко, и подсказать это компилятору через встроенные команды, такие как \texttt{[[likely]]}, за счет чего переключение контекста будет вынесено в <<медленный>> путь и не будет мешать кэшированию инструкций.

Таким образом, \texttt{Marker} может быть переименован в \texttt{MarkerImpl} с минимальными изменениями и без ухудшений производительности в случае, когда с данным сборщиком мусора может исполняться только один язык.
\begin{figure}[!h]
\caption{Модификация иерархии маркеров}\label{fig:gc-markers}
\centering
	\MyResize{\textwidth}{
		\begin{tabular}{ |c|c| }
			\hline
			Старая иерархия & Новая иерархия \\
			\hline
			\includesvg{build/res/dot/marker_was.dot} & \includesvg{build/res/dot/marker_new.dot} \\
			\hline
		\end{tabular}
	}
\end{figure}

\subsection{Обход стеков}
Так как представление объектов на стеке зависит только от метода, для которого известно динамический ли он, и который хранится в каждом стековом кадре, модификации алгоритма сканирования не требуются. Однако, так как два состояния потока находятся на одном реальном стеке, необходимо проверить кто из них сейчас активен (и является самым верхним). Это можно сделать за счет той переменной, добавленной для поддержки консистентности регистра потока и переменной: если значение переменной совпадает с исследуемым потоком, то он активен. Поскольку стек обходиться только во время фаз остановки мира (таких как <<initial mark pause>> и <<remark pause>>), гонки данных не появляется.
\subsection{Перемещение объектов}
На перемещение объектов влияет факт добавления нового объектного тега в NaN упаковку на динамически типизированной стороне. Из-за этого перемещающий код не может копировать значения один в один, как это было раньше, а должен сохранять их тег из предыдущего значения. Данное изменение влечет лишь пару битовых операций.


\chapterconclusion
Таким образом, была описана большая часть изменений, необходимых для взаимодействия двух виртуальных машин, чтобы управление было частично общим и позволяло сохранить преимущества каждой из реализаций.

Можно заметить, что в области межъязыкового взаимодействия есть некоторые крайне трудно решаемые задачи для решения которых порой приходиться идти на компромисс между производительностью и удобством использования.

\chapter{Общий интерфейс межъязыкового взаимодействия}\label{ch:architecture}
Для того, чтобы описанные в предыдущей Главе изменения заработали, необходимо добавить работу с ними в процесс исполнения кода. Общая схема могла бы быть представлена, как <<интерпретатор $\rightarrow$ встраиваемые кэши $\rightarrow$ скомпилированный код>>, так как встраиваемые кэши так же представляют профиль исполнения.

Каждая виртуальная машина снабжается интерфейсом межъязыкового взаимодействия, через который можно в частности получить дополнительные узлы исполнения общих действий, таких как чтение значения по имени. Эти узлы могут быть как аллоцированы для использования во встраиваемых кэшах, так и созданы локально на стеке. Разделение на такие подходы позволяет избежать аллокаций памяти и сбора дополнительного профиля в ситуациях, когда данный интерфейс будет сразу же удален.

Стоит отметить, что так как данная работа фокусируется на изменениях среды исполнения, а большая часть статически типизированных языков полагается на синтаксический сахар или специальное поведение компилятора для конкретных типов или конструкций, эта часть будет освещена довольно поверхностно, покрывая только необходимые <<примитивы>>, но не их синтаксическое представление в исходном коде. Полученные конструкции будут во многом аналогичны динамически типизированной стороне, однако большая часть реализации должна находиться в компиляторе, что выходит за рамки данной работы.

Каждый узел должен обладать следующим набором методов:
\begin{itemize}
	\item создание, принимающее интерфейс создающего языка;
	\item интерпретация соответствующих методов;
	\item компиляция интерпретируемых методов.
\end{itemize}

На схеме взаимодействия на рисунке~\ref{fig:create-interop-node} продемонстрирована примерная схема работы: первая виртуальная машина (VM1) при обнаружении инородного объекта обращается ко второй виртуальной машине (VM2) с целью получения конкретного узла (на схеме обозначено зеленой стрелкой). За счет того, что в момент создания узлу доступны обе виртуальные машины, появляется общая система типов, и осуществляется взаимодействие в обе стороны: чтение отсутствующего аттрибута может быть перенаправлено назад в язык, за счет чего обеспечится семантика вызывающего языка: поднимется исключение или будет возвращено специально значение, например, \texttt{undefined}. Затем для исполнения виртуальная машина обращается к этому узлу (красная стрелка на схеме), который может быть сохранен во встраиваемых кэшах или <<вызываемой стороне>> полиморфного вызова.
\begin{figure}[!h]
	\caption{Схема работы узла межъязыкового взаимодействия}\label{fig:create-interop-node}
	\centering
	\includesvg{build/res/dot/architecture-sample.dot}
\end{figure}

Можно заметить, что при данном подходе необходимо реализовать только необходимые действия, такие как нахождение сущностей (полей и методов) по имени и конверсии типовой системы, проходящие через некоторый установленный набор типов.

\section{Конверсии типов}
Важно сохранение семантики обоих языков, для чего взаимодействие происходит через набор некоторых примитивных типов и два вида объектных. Динамически типизированный язык в такой ситуации может реализовывать сильную или слабую динамическую типизацию, а статически типизированный язык сохранять правила конверсии. Например, конверсия из \texttt{float64} в \texttt{int8} может быть сделана несколькими способами: при помощи усечения, как в C++ (инструкция \texttt{cvttsd2si} на процессоре x86-64), или с семантикой аналогичной Java: сначала происходит преобразование к \texttt{int32} со специальными правилами (например, если значение числа с плавающей точкой больше максимального допустимого, то результат считается равным второму), и затем полученный \texttt{int32} конвертируется к числу меньшей битности~\cite{java8-spec}.

\section{Вызов методов на статически типизированной стороне}
Для вызова методов со статически типизированной стороны подошла бы инструкция \Indy, которая бы позволяла передать сигнатуру, содержащую типы аргументов, известные в точке вызова, и передавала управление узлу межъязыкового взаимодействия. Однако, данная инструкция отсутствует в данный момент на платформе, из-за чего кастомизация за счет узлов невозможна и осуществляется при помощи конечного числа перегрузок.

Может понадобится определение языка, отвечающего за конверсии или конкретной виртуальной машины данного языка, если несколько машин для одного языка могу сосуществовать в одном реальном потоке. Этого можно достигнуть несколькими способами: либо заворачивать все объекты в дополнительную прослойку, которая будет знать о привязанной виртуальной машине; либо совершать все взаимодействие через переменную, предоставляющую к ней доступ. Второй способ является более производительным, чем первый, поскольку не требует дополнительных выделений памяти, но усложняет добавление синтаксического сахара для работы с динамическими объектами, возможно добавление некоторого аналога \texttt{with} из языка программирования Kotlin. Так же возможны оптимизации использующие предположения о том, что в каждой конкретной точке вызова виртуальна машина редко изменяется.

В программной платформе Electron в целях безопасности разделяются контексты языка ecmascript, отвечающие за логику, и отвечающие за графическое взаимодействие~\cite{electron-isolates}.

\section{Вызов методов на динамически типизированной стороне}
Наибольшую сложность для вызова методов на динамически типизированной стороне представляет перегрузка функций и методов. Это видно и по тому, разрешают ли ее наиболее популярные языки:
\begin{itemize}
	\item не разрешают: Lua, Python~\cite{overload-python}, Ruby, Ecmascript;
	\item разрешают по числу аргументов: Elixir~\cite{overload-elixir}, Erlang (и Prolog);
	\item разрешают несколько сигнатур: Typescript~\cite{overload-typescript}.
\end{itemize}

Перегрузка по числу аргументов зачастую разрешается в точке вызова на этапе компиляции или преобразования в байткод за счет изменения имени, в остальных случаях языки обязывают иметь одно общее тело функции и выбирать путь через динамические конструкции такие как \texttt{typeof}. Возможно собрать профиль типов аргументов в точке вызова, однако это замедлит выполнение в случае, когда межъязыковое взаимодействие не используется, без особых преимуществ, в то время как хотелось бы следовать принципу <<not pay for what you don't use>>\footnote{англ. не плати за то, что не используешь}. Таким образом, наиболее выгодная стратегия это найти наиболее общую перегрузку, или, если ее не существует, вернуться к исходному уровню компиляции, который переберет все подходящие.

Однако, GraalVM вызывает самую специфическую версию, т.е. выражение вызова функции \texttt{f(1)} вызывает перегрузку \texttt{f(int8)}, а не \texttt{f(int32)}, поскольку тип \texttt{int8} меньше типа \texttt{int32}. На первый взгляд может показаться, что такой подход лучше, однако он имеет значительные недостатки:
\begin{enumerate}
	\item Численные литералы по-разному обрабатываются в языках программирования: если в какой-то степени поддерживается вывод типов (в случае чего язык почти наверняка не поддерживает уже перегрузку), литерал скорее всего представляется как \texttt{intSome}, например \texttt{fromInteger 1 :: Num} в языке программирования Haskell, или \texttt{comptime\_int} в Zig'е. Этот тип затем может быть преобразован к нужному во время решения системы уравнений о типах. В Java, $1$ представляет собой литерал типа \texttt{int}, а значит \texttt{f(1)} вызовет именно \texttt{f(int32)}, а не от более узкого типа, что означает, что при простом переносе строчки из статически типизированного языка в динамически типизированный изменилась семантика.
	\item Нахождение <<самой конкретной специализации>> означает добавление нескольких проверок во время исполнения, которых невозможно избежать за счет сбора профиля, поскольку даже если он говорит, что число поместиться в \texttt{int16}, это необходимо проверить, и в добавок к этому убедиться, что оно не помещается в \texttt{int8}. Если в языке присутствуют типы без знака, то картина усложняется еще сильнее, и становится совсем не понятно что выбирать, например, число $16$ должно вызвать функцию от целого числа без знака или со знаком? Если же на вход подается число с плавающей точкой, также необходимо проверить является ли оно целым.
	\item Такой вызов перегрузки в некотором смысле навязывает динамически типизированному языку чужую типизацию, ведь если в его системе есть только один численный тип~--- число с плавающей точкой двойной точности, то выглядит логичным вызвать перегрузку именно от него.
\end{enumerate}

При всем выше сказанном перегрузки по большей части <<прозрачны>>: программисту не важно вызовется \texttt{print(int64)} или \texttt{print(int32)}, так как семантически они делают одно и то же.

\section{Аналогии с другими реализациями}
Обобщая сказанное в Главах~\ref{ch:platform} и \ref{ch:architecture}, можно заметить некоторую аналогию с мета-трассировкой, однако узлы межъязыкового взаимодействия локализованы именно для этого взаимодействия, что позволяет сохранить преимущества платформы, при этом оставляя реализацию межъязыкового взаимодействия относительно простой и гибкой. В прототипе данные узлы и интерфейс реализованы на C++, поскольку их реализация невозможна на текущих управляемых языках платформы в частности потому, что требовала бы переключения состояния потока и не имела бы доступа к таким низкоуровневым операциям, как конвертация соглашений о вызовах.

Создание узлов межъязыкового взаимодействия и взаимодействие с ними может рассматриваться как передача сообщений между виртуальными машинами~\cite{graalvm-polyglot}. За счет наличия общего промежуточного представления в компиляторе данные сообщения могут быть конвертированы в эффективный код с некоторыми конверсиями или упаковками в динамические значения на границах сообщений-инструкций. За счет последующих оптимизаций (которые аналогичны описанным в Разделе~\ref{ch:work-with-fields}) взаимообратные конструкции могут быть удалены, как это происходит c оптимизацией операций над списками и другими потоками данных, например, в языке Haskell~\cite{streamfusion}, что позволяет перейти к межъязыковым оптимизациям, которые недоступны при использовании FFI из-за отсутствия информации об исходном коде и невозможности его модифицировать.

\section{Особенности предложенного подхода}\label{sec:peculiarities}
Предложенная реализация обладает несколькими особенностями:
\begin{itemize}
	\item требуются аллокации узлов исполнения, из-за чего они расположены в адресах далеких от профиля, из-за чего доступ к памяти становится менее оптимальным с точки зрения промахов по кэшу. В результате чего они служат скорее для сбора профиля, чем для улучшения производительности;
	\item требуется индирекция при вызове; причем для обработки одного узла их может потребоваться несколько: например, при чтении статического поля его необходимо <<завернуть>> в динамическое значение, чем может заниматься специальный объект, предоставленный из интерфейса динамического языка. Важно заметить, что количество статических типов ограничено, а операция не требует сбора профиля, поскольку тип известен статически, что означает, что все такие <<заворачиватели>> могут быть выделены в памяти заранее и не обладать никакими условными проверками внутри, что позволит терять меньшее количество производительности.
\end{itemize}

У реализации на C++ есть ряд дополнительных недостатков, свойственных нативному коду:
\begin{itemize}
	\item ручное управление памятью, реализованное через подсчет ссылок;
	\item необходимость ввода ограничений на методы, которые могут вызывать сборщик мусора;
	\item необходимость аллокации некоторых данных как не перемещаемых.
\end{itemize}

\section{Общие проблемы межъязыкового взаимодействия}
В данном разделе обозреваются общие проблемы межъязыкового взаимодействия, большую часть из которых данная работа не стремится решить в общем виде, однако не будет полной без их обзора.

\subsection{Строки}
Так как представление строк в памяти различно, необходимо копирование. Благодаря неизменяемости в большинстве языков, можно было бы добавить указатель из динамически типизированной строки к обычной, который затем лениво инициализировать. Обратная трансформация не может быть осуществлена из-за многопоточности и того, что статический язык может быть связан с несколькими динамическими. Однако, благодаря тому, что это инородный объект, многие методы будут все так же доступны, но без специального синтаксиса, такого как оператор плюс.

В реализованном прототипе вводится конверсия статически типизированной строки к динамически типизированной, однако она происходит за счет копирования.

\subsection{Массивы}
% https://wiki.python.org/jython/CollectionsIntegration
Массивы сталкиваются с проблемами аналогичными строкам: у них разные типы. Однако, динамическому языку это может быть не заметно благодаря <<утиной>> типизации, если языко-специфичный интерфейс предоставлен наряду с обычным. С другой стороны, возникают проблемы с вариантностью для типизированного массива, поскольку эта информация может теряться на уровне байткода. Другие реализации, такие как jython, запрещают конверсии массивов.

В предложенной реализации на платформе Ark статически типизированные массивы на динамически типизированной стороне обладают только доступом по индексу и свойством длины.

\subsection{Возвращение экземпляров примитивных типов из динамически типизированных функций}
В данной области популярным подходом являлось бы возвращение запакованного значения соответствующего примитивного типа (например, \texttt{Double} для \texttt{double} в Java), однако это требовало бы аллокацию и более того корректной проверки в точке приема: сначала проверки на то, статический объект или динамический, и затем все дальнейшие проверки для статического объекта перед использованием или передачей назад в динамически типизированную функцию. Однако, если разрешить тип \texttt{any} на статически типизированный стороне с минимальным набором действий, таких как присвоение в переменную или передача как аргумента в функцию, то будет возможным реализовать унифицированный интерфейс межъязыкового взаимодействия, при котором полученное из динамически типизированной функции значение можно вернуть назад без дополнительных проверок. В данный момент над таким функционалом ведется работа и динамические функции возвращают лишь объектные типы.

\subsection{Наследование}
Наследование было бы возможно реализовать за счет генерации статического класса, наследующего данный с дополнительным полем: динамическим объектом (\texttt{DynamicObject}), который бы переопределял все методы следующим образом: сначала проверить есть ли поле с соответствующем именем внутри того объекта, и если нет, то вызвать родительский (или вызвать исключение о неопределенном абстрактном методе). Данный механизм в прототипе не был реализован, поскольку наилучшим образом реализуется через аналог \texttt{DynamicProxy} из языка Java, отсутствующего на платформе в данный момент.

%\subsection{Null}
%Внимательный читатель мог заметить, что расширение системы типов новым тегом на динамически типизированной стороне не прошло незамеченным: появился еще один \texttt{null}, что указывает на некоторые недостатки использования NaN упаковки вместо способа, который использует GraalVM: простой запаковки примитивов в их объектные аналоги.\\

%Помимо этого, существует несколько способов <<борьбы>> с пустыми значениями из динамического языка (например, во что должен превращаться \texttt{undefined}?): они все могут преобразовываться к \texttt{null}, либо быть представлены аналогично \texttt{Unit} из языка Kotlin: константными ссылочными значением. Первый способ не уступает второму, поскольку \texttt{undefined} чаще всего означает отсутствие возвращаемого значения, а, например, в Java, запакованный тип \texttt{Void} представляется единственным значением~--- \texttt{null}~\cite{java-lang-void}.

\chapterconclusion
Не смотря на перечисленные в Разделе~\ref{sec:peculiarities} особенности, реализованный подход позволяет унифицировать интерфейс межъязыкового взаимодействия с сохранением точек кастомизации, учитывающих семантики обоих языков.

Улучшение предлагаемого подхода заключается в наличии некоторого специального языка, который с одной стороны позволял бы совершать низкоуровневые операции, а с другой давал бы контролируемый доступ к управляемым конструкциям, таким как безопасные состояния. Наличие дополнительного языка позволяет упростить встраивание, поскольку избавляет от необходимости отдельно писать код, компилирующий узел и отдельно интерпретирующий. Схожая техника используется для упрощения разработки, например в v8 с языком torque \cite{torque}, который тем не менее решает другую задачу, но так же позволяет интегрировать конструкции, не представимые на целевом языке v8 (ecmascript) с сохранением производительности.

Предлагаемые требования, которые можно было бы выдвинуть к промежуточному языку:
\begin{enumerate}
	\item независимость от текущего состояния потока;
	\item возможность выполнения базовых операций всех языков;
	\item доступ к низкоуровневым операциям и конверсии соглашений о вызовах, смене состояния потока;
	\item конвертация во внутренне представление;
	\item автоматическое управление объектами.
\end{enumerate}

Таким образом, вместе с изменениями платформы данная архитектура позволяет скомпилированному коду на динамически типизированной стороне взаимодействовать с конструкциями статически типизированного языка без интринсиков или нативного кода после JIT компиляции, что удовлетворяет требованиям работы. Улучшения производительности со стороны статически типизированного языка находятся в разработке, поскольку на платформе в данный момент нет никакого аналога \Indy, без использования которого ускорение взаимодействия не представлялось возможным ввиду динамической типизации.

\chapter{Оценка полученного прототипа}\label{ch:results}
В качестве метрики объема проделанной работы могут выступать строки кода. Без учета кода тестов, было добавлено более 7500 строк кода, а удалено менее 1000. Результаты же оценивались при помощи тестирования и замеров производительности.

\section{Методы тестирования}
Оценка корректности производилась за счет сравнения наблюдаемого поведения на наборе тестов с ожидаемым и верификацией, что интерпретируемый код дает тот же результат, что и скомпилированный, для чего JIT компиляция делалась синхронной. Во время тестирования отслеживались события, такие как деоптимизация, для проверки, что код инвалидируется при нарушении предположений и не происходит чтений неправильных адресов, а также для того, чтобы протестировать, что скомпилированный код исполняется, а не сразу же деоптимизируется.

Для проверки работоспособности оптимизаций некоторые тесты проверяли число тех или иных инструкций после определенных проходов оптимизатора по промежуточному представлению JIT компилятора.

\section{Замеры производительности}\label{sec:sunspider}
\subsection{Описание кода}
Замер производился на коде, полученном путем модификации общепринятого теста <<Sunspider/AccessNBody>>, входящего в набор замеров производительности разработанных командой WebKit для тестирования реализаций языков удовлетворяющих открытому стандарту ecmascript. Данный тест не является полностью синтетическим, поскольку просчитывает положение тел в солнечной системе, крайне упрощенная версия которого представлена на рисунке~\ref{bench-access-code}.

\begin{figure}
\centering
\caption{Упрощенный код теста Sunspider/AccessNBody}\label{bench-access-code}
\begin{tabular}[t]{p{0.5\textwidth}|p{0.5\textwidth}}
\begin{lstlisting}
class System {
	class Body { x: double }
	advance(t: double): void
	bodies: Array<Body>
}
\end{lstlisting} &
\begin{lstlisting}
function System.advance(t) {
	// for i { for j {
	this.bodies[i].x +=
		this.bodies[j].x * t
	// } }
}
\end{lstlisting}
\end{tabular}
\end{figure}

Т.е. в цикле двойной вложенности происходят действия над массивом <<структур>>. Производительность сравнивается в трех модификациях, расположенных слева направо на каждой платформе и соединенных стрелками:
\begin{enumerate}
	\item сначала все объекты представлены динамически типизированными (<<динамические объекты>>);
	\item затем элементы массива (экземпляры типа \texttt{Body}) заменяются на статически типизированные (<<статический \texttt{Body}>>);
	\item затем весь массив становится статически типизированным (<<статический \texttt{Array<Body>}>>).
\end{enumerate}

Также важно заметить, что программа написана на динамически типизированном языке и полностью мономорфна, т.е. при каждом вызове каждой функции типы аргументов (в том числе динамические классы) совпадают.

\begin{figure}[!h]
	\caption{Производительность интерпретации Sunspider/AccessNBody}\label{fig:accessnbodyinterpres}
	\resizebox{\textwidth}{!}{\includesvg{build/res/data-build/access-nbody-interpreter.svg}}
\end{figure}

Первая проверка производилась при выключенном компиляторе для демонстрации преимуществ платформы Ark, указанных в Разделе~\ref{ch:ark-good}. График времени исполнения тестов на различных платформах представлен на рисунке~\ref{fig:accessnbodyinterpres} (меньшие значения отвечают лучшим результатам). Важно заметить, что GraalVM не позволяет отключить компилятор, из-за чего график ее производительности некорректен: он должен был быть расположен выше, поскольку часть кода была JIT скомпилирована (возможно, код интерпретатора дерева). По этой же причине она представлена лишь одной точкой. Однако, не смотря на вышесказанное, данный график все еще демонстрирует преимущества наличия нескольких вирутальных машин.

Результаты второго, более интересного в рамках оценки данной работы, а не платформы Ark в целом, скомпилированного замера производительности представлены на рисунке~\ref{fig:accessnbodyres}. Полные результаты, включающие информацию о процессоре, на котором выполнялись замеры и исходный код модификаций описаны в приложении~\ref{ap:sunspider}.
\begin{figure}[!h]
	\caption{Результаты замеров Sunspider/AccessNBody}\label{fig:accessnbodyres}
	\resizebox{\textwidth}{!}{\includesvg{build/res/data-build/access-nbody.svg}}
\end{figure}

Точки выброса, вероятно, связаны с тем, что замеры производились с включенным сборщиком мусора. Программа выполнялась порядка тысячи раз без перезапуска для достижения JIT компиляции; на графике~\ref{fig:accessnbodyres} представлены последние 20\% запусков.

\subsection{Результаты на платформе Ark}
Результаты на платформе Ark оказались интерпретируемыми:
\begin{itemize}
\item при первом преобразовании стало на одну динамическую проверку меньше: чтение поля \texttt{Body.x} лишилось проверки на принадлежность числу, поскольку его тип заранее известен;
\item при втором же все проверки типов внутри цикла заменяются на простую проверку на нулевой указатель, поскольку известно какому классу принадлежат элементы; за счет вида цикла проверки на выход за рамки массива так же отсутствуют.
\end{itemize}

Это и привело к последовательному улучшению производительности примерно в два раза.

\subsection{Результаты на других платформах}
Результаты на других платформах оказались куда менее очевидными, что, вероятно, связано с тем, что узлы интерпретатора (реализованные через \Indy в Nashorn) не адаптируются под объекты статически типизированного языка. Однако общая картина показывает, что платформа Ark в целом, и межъязыковое взаимодействие на ней в частности, являются конкурентноспособными в случае скомпилированного кода, и поигрывают node (v8) в то же число раз, что и альтернативные реализации.

\chapterconclusion
Полученный в результате выполнения работы прототип не только показал свою работоспособность в примитивных сценариях, проверяемых тестами, но и позволил сократить время исполнения общепринятого теста за счет добавления статической типизации объекту, что является демонстрацией применимости межъязыкового взаимодействия и в обычных программах.

\startconclusionpage
В ходе выполнения работы были получены следующие результаты:
\begin{enumerate}
	\item был проведен обзор области, покрывающий многочисленное число существующих подходов в области межъязыкового взаимодействия для различных языков и платформ;
	\item были выявлены недостатки существующих методов, находящихся в той же классификации по свойству управляемости взаимодействующих языков, что и платформа Ark;
	\item было предложено решение для каждой компоненты виртуальной машины, участвующей в процессе, включая управление памятью, интерпретатор и компилятор;
	\item различные вирутальные машины получили возможность исполнять код на одном нативном стеке и использовать общий сборщик мусора;
	\item был предложен и реализован высокоуровневый интерфейс межъязыкового взаимодействия, позволяющий эффективно JIT компилировать межъязыковое взаимодействие при обращении динамически типизированной стороны к статически типизированной;
	\item код, демонстрирующий применимость предложенного подхода, был написан и опубликован;
	\item была проведена оценка результатов при помощи тестов и замеров производительности, которая показала прирост скорости в два раза на общепринятом тесте за счет замены некоторых объектов на статически типизированные;
	\item были предложены дальнейшие улучшения.
\end{enumerate}

Дальнейшая работа в описанной области должна рассматривать применимость дополнительного языка для описания узлов исполнения межъязыкового взаимодействия. Возможно исследование механизмов, позволяющих ускорить межъязыковое взаимодействие, направленное из статически типизированного языка в динамически типизированный, альтернативных подходу использования аналогов инструкции \Indy из JVM.

Исходный код прототипа доступен по следующим ссылкам:
\begin{itemize}
	\item \url{https://gitee.com/kprokopenko/arkcompiler_runtime_core/tree/interop-mem/}
	\item \url{https://gitee.com/kprokopenko/arkcompiler_ets_runtime/tree/interop-mem/}
\end{itemize}

\printmainbibliography
\appendix

\chapter{Листинги}
\begin{figure}[!h]
	\caption{Микро замер производительности вызова функции из Python с и без применения cppyy}\label{apx:cppyy-bench}
	%\centering
	\lstinputlisting[language=Python]{../data/cppyy-sample.py}
\end{figure}

\chapter{Данные о замере производительности Sunspider/AccessNBody}\label{ap:sunspider}
Использованные репозитории и версии кода:
\lstinputlisting[language=none]{../data/bachelor-thesis-info-mirror/artifacts/repo.info}

Данные процессора, запускающего код:
\lstinputlisting[language=none]{../data/bachelor-thesis-info-mirror/artifacts/cpu.info}

Обновленные данные и исходный код доступны по ссылке: \url{https://github.com/kp2pml30/bachelor-thesis-info-mirror/tree/artifacts}

\end{document}
