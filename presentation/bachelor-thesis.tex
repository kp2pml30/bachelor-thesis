\documentclass[xcolor=table, aspectratio=169]{beamer}

\usepackage{fontspec}
\defaultfontfeatures{Ligatures={TeX},Renderer=Basic}
\setmainfont[Ligatures={TeX,Historic}]{Times New Roman}
\setsansfont{Arial}
\setmonofont{Courier New}

\usepackage[english,russian]{babel}
\usepackage{svg}
\usepackage{pifont}
\usepackage{tikz}
\usepackage{ifthen}
\usepackage{ulem}
\usepackage{listings}
\usepackage{comment}
\usepackage{ITMOtheme}

\def\MyResearchTitle{Поддержка вызова функций и передачи объектов между реализациями языков со статическими и динамическими типизациями в среде выполнения на платформе Ark}

\def\MyResearchAim{Разработка межъязыкового взаимодействия на платформе Ark между реализациями языков со статическими и динамическими типизациями с минимизацией неуправляемого кода}

\def\MyResearchTargets{
	\MyResearchTargetNext{обзор существующих подходов в области межъязыкового взаимодействия}{;}
	\MyResearchTargetNext{выделение затрагиваемых компонент платформы и их архитектурных ограничений}{;}
	\MyResearchTargetNext{проектирование интерфейса межъязыкового взаимодействия}{;}
	\MyResearchTargetNext{реализация прототипа, минимизирующего количество неуправляемого кода}{;}
	\MyResearchTargetNext{оценка результатов с точки зрения корректности и производительности}{.}
}

\lstdefinelanguage{etspseudo}{
	keywords={let, const, typeof, new, true, false, catch, function, return, null, catch, switch, var, if, in, while, do, else, case, break},
	keywordstyle=\color{blue}\bfseries,
	ndkeywords={class, export, boolean, throw, implements, import, this, void, int, long, short, double},
	ndkeywordstyle=\color{purple}\bfseries,
	identifierstyle=\color{black},
	sensitive=false,
	comment=[l]{//},
	morecomment=[s]{/*}{*/},
	commentstyle=\color{purple}\ttfamily,
	stringstyle=\color{red}\ttfamily,
	morestring=[b]',
	morestring=[b]"
}


\title[Межъязыковое взаимодействие на платформе Ark]{\MyResearchTitle}
\ifthenelse{\equal{\presMode}{full}}{
	\author{Прокопенко К.Д.}
	\institute{Open Harmony}
	} {
	\author{Студент: Прокопенко К.Д., M34361\\Руководитель: Беляев Е.А., к.т.н., доцент ФИТиП\\Консультант: Соломенников Д.И.}
	\institute{ИТМО\\бакалаварская работа}
}
\date{2023}

\newcommand{\cmark}{\ding{51}}
\newcommand{\xmark}{\ding{55}}
\newcommand{\MyDone}{{\color{ITMOred}\cmark}}

\let\SlidesAtEnd\empty

\ifthenelse{\equal{\presMode}{full}}{%
\newcommand{\MoveSlideToEnd}[1]{#1}%
} {%
\newcounter{slidesAtEnd}
\newcommand{\MoveSlideToEnd}[1]{
	\stepcounter{slidesAtEnd}
	\expandafter\let\csname OldSlidesAtEnd\theslidesAtEnd\endcsname\SlidesAtEnd
	\expandafter\def\csname SlideContent\theslidesAtEnd\endcsname{#1}
	\edef\Tmp{\expandafter\noexpand\csname OldSlidesAtEnd\theslidesAtEnd \endcsname\expandafter\noexpand\csname SlideContent\theslidesAtEnd \endcsname}
	\let\SlidesAtEnd\Tmp
}%
}

\newcommand{\BackupBegin}{
	\newcounter{finalframe}
	\setcounter{finalframe}{\value{framenumber}}
}
\newcommand{\BackupEnd}{
	\setcounter{framenumber}{\value{finalframe}}
}

\makeatletter
\def\rowcolor{{\ifnum0=`}\fi\bmr@rowcolor}
\newcommand<>{\bmr@rowcolor}{%
	\alt#1%
		{\global\let\CT@do@color\CT@@do@color\@ifnextchar[\CT@rowa\CT@rowb}%
		{\ifnum0=`{\fi}\@gooble@rowcolor}%
}
\newcommand{\@gooble@rowcolor}[2][]{\@gooble@rowcolor@}
	\newcommand{\@gooble@rowcolor@}[1][]{\@gooble@rowcolor@@}
	\newcommand{\@gooble@rowcolor@@}[1][]{\ignorespaces}
\makeatother

\setfootlinetext{\quad\insertsection\ifthenelse{\equal{\insertsubsection}{}}{}{ / \insertsubsection}}

\lstdefinelanguage{etspseudo}{
	keywords={let, const, typeof, new, true, false, catch, function, return, null, catch, switch, var, if, in, while, do, else, case, break},
	keywordstyle=\color{blue}\bfseries,
	ndkeywords={class, export, boolean, throw, implements, import, this, void, int, long, short, double},
	ndkeywordstyle=\color{purple}\bfseries,
	identifierstyle=\color{black},
	sensitive=false,
	comment=[l]{//},
	morecomment=[s]{/*}{*/},
	commentstyle=\color{purple}\ttfamily,
	stringstyle=\color{red}\ttfamily,
	morestring=[b]',
	morestring=[b]"
}

\lstset{
	showtabs=true,
	breaklines=true,
	language=etspseudo,
}

\makeatletter
\patchcmd{\beamer@sectionintoc}
	{\vfill}
	{\vskip-\itemsep}
	{}
	{}
\makeatother

\def\MyResearchTargetNext#1#2{\item {\MakeUppercase#1}}

\begin{document}

\tikzstyle{every picture}+=[remember picture]
\tikzstyle{na}=[shape=rectangle,inner sep=0pt,text depth=0pt]

\begin{comment}

\AtBeginSection[]
{
	\begin{frame}[plain, noframenumbering]
		\frametitle{Outline}
		%\Large
		\small
		\tableofcontents[currentsection, subsectionstyle=show/show/hide]
	\end{frame}
}

\AtBeginSubsection[]
{
	\begin{frame}[plain, noframenumbering]
		\frametitle{Outline}
		\small
		\tableofcontents[currentsection, currentsubsection, subsectionstyle=show/shaded/hide]
	\end{frame}
}

\end{comment}

\begin{frame}[plain]
	\titlepage
\end{frame}

\ifthenelse{\equal{\presMode}{full}}{
} {
	\begin{frame}[plain, noframenumbering]
		\frametitle{Ссылки на материалы работы}
		\begin{itemize}
			\item презентация
				\begin{itemize}
					\item\url{https://github.com/kp2pml30/bachelor-thesis}
				\end{itemize}
			\item реализация
				\begin{itemize}
					\item\url{https://gitee.com/kprokopenko/arkcompiler_runtime_core/tree/interop-mem/}\\
					\item\url{https://gitee.com/kprokopenko/arkcompiler_ets_runtime/tree/interop-mem/}
				\end{itemize}
		\end{itemize}
	\end{frame}
	\addtocounter{framenumber}{-1}
}

\section{Обзор}

\def\HalfPageSize{0.4\textwidth}
\newcounter{actualListNum}

\begin{frame}
\frametitle{Актуальность межъязкового взаимодействия}
\begin{center}
\begin{tabular}{p{\HalfPageSize}|p{\HalfPageSize}}
Для программистов&Для компаний\\
\hline\\
\begin{enumerate}
\item Переиспользование кода
\item Миграция на другой язык
\item Выбор подходящего языка
\item Использование экспериментальных решений
\setcounter{actualListNum}{\value{enumi}}
\end{enumerate}
&
\begin{enumerate}
\setcounter{enumi}{\value{actualListNum}}
\item Скорость разработки
\item Упрощение найма сотрудников
\end{enumerate}
\end{tabular}
\end{center}
\end{frame}

\begin{frame}
\frametitle{Цель и задачи}
\textbf{Цель:} \MyResearchAim\\
\textbf{Задачи:}
\begin{itemize}
	\MyResearchTargets
\end{itemize}
\end{frame}

\begin{frame}
\frametitle{Существующие реализации межъязыкового взаимдоействия}
\begin{columns}[T]
\begin{column}{0.5\textwidth}
\begin{itemize}
	\item Нативный-нативный
		\begin{itemize}
			\item{C++, Zig, Carbon}
		\end{itemize}
	\pause
	\item Управляемый-нативный
		\begin{itemize}
			\item{FFI (Haskell, python, \dots)}
			\item{JNI, Project Panama}
			\item{LuaJIT FFI}
			\item{CPPYY}
		\end{itemize}
	\pause
	\item[\cmark] Управляемый-управляемый
		\begin{itemize}
			\item{JVM/.NET}
			\begin{itemize}
				\item{Kotlin, Scala}
				\item[\cmark]{Nashorn}
				\item[\cmark]{Graal VM}
			\end{itemize}
		\end{itemize}
\end{itemize}
\end{column}

\begin{column}{0.5\textwidth}
\pause
Недостатки существующих решений~({\color{ITMOred}\cmark})
\begin{enumerate}
	\item Наследование ограничений целевой виртуальной машины
	\item Неоптимальность интерпретации
	\item Увеличенное время на разогрев
\end{enumerate}
\end{column}

\end{columns}
\end{frame}

\begin{frame}
\frametitle{Чем отличается платформа Ark?}
Она объединяет в себе несколько виртуальных машин, что влечет следующее:
\begin{enumerate}
	\item Различные среды исполнения
	\item Различные наборы инструкций
	\item Различное представление данных
	\item Полное отсутствие межъязыкового взаимодействия
\end{enumerate}
Однако, реализации имеют общие компоненты; и задача разбивается на части:
\begin{enumerate}
	\item Изменения платформы
	\item Создание общего интерфейса межъязыкового взаимодействия
\end{enumerate}
\end{frame}

\section{Изменения платформы}

\begin{frame}
\frametitle{Затрагиваемые компоненты}
\begin{center}
\resizebox{!}{0.8\textheight}{
\includesvg{build/res/dot/components.dot}
}
\end{center}
\end{frame}

\subsection{Представление объектов}

\begin{frame}
\frametitle{Представление объектов на динамически типизированной стороне~--- NaN упаковка}
Представление чисел с плавающей точкой двойной точности
\begin{center}
\begin{tabular}{|c|c|c|c|c|c|}
\hline
число бит & 1 & 11 & \multicolumn{3}{c|}{52} \\
\hline
значение & знак & экспонента & \multicolumn{3}{c|}{мантисса} \\
\hline
NaN & знак & 1\dots 1 & \multicolumn{3}{c|}{мантисса} \\
\hline
qNaN indefinite & знак & 1\dots 1 & 1 & \multicolumn{2}{c|}{0\dots0} \\
\hline
кодирование & знак & 1\dots 1 & 1 & тэг & значение \\
\hline
\end{tabular}
\end{center}
\end{frame}

\MoveSlideToEnd{

\begin{frame}
\frametitle{Представление объектов на динамически типизированной стороне~--- NaN упаковка (Ark)}
\begin{center}
\begin{tabular}{|c|c|c|}
\hline
число бит & 16 & 48 \\
\hline
значение & ``тэг'' & нагрузка \\
\hline
object & \texttt{0000} & указатель \\
\hline
float64 & \begin{tabular}{c} \texttt{0001} \\ --- \\ \texttt{fffe} \end{tabular} & \dots \\
\hline
int32 & \texttt{ffff} & int32 \\
\hline
\end{tabular}
\end{center}
\end{frame}

\begin{frame}
\frametitle{Представление объектов на динамически типизированной стороне~--- NaN упаковка (Ark)}
\begin{center}
\begin{tabular}{|c|c|c|c|}
\hline
число бит & 16 & \multicolumn{2}{c|}{48} \\
\hline
значение & ``тэг'' & \multicolumn{2}{c|}{нагрузка} \\
\hline
object & \texttt{0000} & \texttt{0} & указатель \\
\hline
foreign object & \texttt{0000} & \texttt{1} & указатель \\
\hline
float64 & \begin{tabular}{c} \texttt{0001} \\ --- \\ \texttt{fffe} \end{tabular} & \multicolumn{2}{c|}{\dots} \\
\hline
int32 & \texttt{ffff} & \multicolumn{2}{c|}{int32} \\
\hline
\end{tabular}
\end{center}
\end{frame}

}

\begin{frame}
\frametitle{Представление объектов}
\begin{center}
\begin{tabular}{b{\HalfPageSize}|b{\HalfPageSize}}
На динамически типизированной стороне&На статически типизированной стороне\\
\hline
\begin{itemize}
	\item[] Как представить статический объект?
	\begin{itemize}
		\item[\cmark] Новым тегом
	\end{itemize}
	\item[] Как представить статические поля и методы?
	\begin{itemize}
		\item[\cmark] Полиморфизмом по типу объекта
	\end{itemize}
\end{itemize}
&\scalebox{0.8}{\includesvg{build/res/dot/statically-typed-tree.dot}}
\end{tabular}
\end{center}
\end{frame}

%\ifthenelse{\equal{\presMode}{full}}{
%	\section{``Присваивающее действие''}
%} {
%}

\subsection{Управление потоками}

\begin{frame}
\frametitle{Состояние потока}
\begin{itemize}
	\item Инкапсулирует данные необходимые для исполнения языка
	\item Хранится в переменной локальной для потока или в регистре
	\item Должно быть консистентно\pause, что означает, что:
	\begin{itemize}
		\item[\cmark] необходимо подменять при вызове функций из другого языка
		\item[\cmark] необходимо возвращать при обработке исключений
		\item[\cmark] необходимо переключать в мостах
		\item[\cmark] не должны появляться гонки данных
	\end{itemize}
\end{itemize}
\ifthenelse{\equal{\presMode}{full}}{\pause{\color{red} Необходимо обмануть \texttt{MutatorLock}}} {}
\end{frame}


\subsection{Управление памятью}

\begin{frame}
\frametitle{Элементы управления памятью}
\begin{enumerate}
	\item Аллокация объектов
	\item Обход стеков\only<2>{ \MyDone}
	\begin{enumerate}
		\item определение типа метода
		\item выбор стеков из состояний потока\only<2>{ \MyDone}
	\end{enumerate}
	\item Маркировка объектов\only<2>{ \MyDone}
	\item Перемещение объектов\only<2>{ \MyDone}
	\ifthenelse{\equal{\presMode}{full}}{
		\pause
		\item Выбор общей настройки сборщика мусора ({\color{red}\texttt{MT\_MODE\_TASK}})
		\item Проблемы с выбором {\color{red}\texttt{ReferenceProcessor}}'а}{}
\end{enumerate}
\end{frame}

\MoveSlideToEnd{
\begin{frame}
\frametitle{Иерархия маркеров --- старая}
\begin{center}
\includesvg{build/res/dot/marker_was.dot}
\end{center}
\end{frame}
\begin{frame}
\frametitle{Иерархия маркеров --- новая}
\begin{center}
\scalebox{0.6}{
\includesvg{build/res/dot/marker_new.dot}
}
\end{center}
\end{frame}
}

\section{Общее устройство межъязыкового взаимодействия}

\begin{frame}
\frametitle{Чего не хватает для исполнения программ?}
\begin{center}
\begin{tabular}{p{0.3\textwidth}p{0.3\textwidth}p{0.3\textwidth}}
\tikz\node[na](archInterp){интерпретатор}; & \tikz\node[na](archIC){встраиваемые кэши}; & \tikz\node[na](archJIT){скомпилированный код};
\begin{tikzpicture}[remember picture,overlay,cyan,rounded corners,>=stealth,shorten > =1pt,shorten <=1pt,thick]
	\draw[->, color=black] (archInterp) to (archIC);
	\draw[->, color=black] (archIC) to (archJIT);
\end{tikzpicture}
\end{tabular}
\end{center}
\pause
Необходим общий интерфейс, который позволял бы:
\begin{itemize}
	\item Создавать узлы взаимодействия
	\item Интерпретировать и компилировать эти узлы
	\item Совершать конверсии типов
\end{itemize}
\end{frame}

\begin{frame}
\frametitle{Схема обращения к объекту другого языка}
\includesvg[height=0.9\textheight]{build/res/dot/architecture-sample.dot}
\end{frame}

\MoveSlideToEnd{
\begin{frame}
\frametitle{Дальнейшие улучшения архитектуры}
Необходимо наличие дополнительного языка\ifthenelse{\equal{\presMode}{full}}{ ({\color{ITMOgold}irtoc?})}{}, обладающего свойствами:
\begin{enumerate}
	\item Независимость от текущего состояния потока
	\item Возможность выполнения базовых операций всех языков
	\item Доступ к низкоуровневым операциям и конверсии соглашений о вызовах
	\item Конвертация во внутренне представление
	\item Автоматическое управление объектами
\end{enumerate}
\end{frame}
}

\section{Оценка результатов}

%\begin{frame}
%\frametitle{Тестирование}
%\begin{itemize}
%	\item Валидация семантики
%	\item Сравнение результата полной интерпретации и JIT компиляции
%	\item Тестирование оптимизаций через промежуточное представление
%\end{itemize}
%\end{frame}

\subsection{Замеры производительности}

\begin{frame}[fragile]
\frametitle{Sunspider/AccessNBody --- Код}
\begin{center}
\begin{tabular}[t]{p{0.45\textwidth}|p{0.55\textwidth}}
\begin{lstlisting}
class System {
  class Body { x: double }
  advance(t: double): void
  bodies: Array<Body>
}
\end{lstlisting} &
\begin{lstlisting}[escapechar=\%]
function System.advance(t) {
  // for i { for j {
  this.bodies[i].x +=
    this.bodies[j].x * t
  // } }
}
\end{lstlisting}
\end{tabular}
\end{center}
\pause
Рассматриваются модификации:
\begin{enumerate}
	\item Код и объекты динамически типизированы
	\item $\rightarrow$ \texttt{Body} статически типизирован
	\item $\rightarrow$ \texttt{Array<Body>} статически типизирован
\end{enumerate}

\end{frame}

\begin{frame}
\frametitle{Sunspider/AccessNBody --- Результаты}
\begin{center}
\includesvg[height=0.8\textheight]{build/res/data-build/access-nbody}
\end{center}
\end{frame}

\begin{frame}[fragile]
\frametitle{Sunspider/AccessNBody --- Код (объяснение)}
\def\MakeMark#1#2{{\color{ITMOpistachio}[#1$\rightarrow$#2]}}
\begin{center}
\begin{tabular}[t]{p{0.45\textwidth}|p{0.55\textwidth}}
\begin{lstlisting}
class System {
  class Body { x: double }
  advance(t: double): void
  bodies: Array<Body>
}
\end{lstlisting} &
\begin{lstlisting}[escapechar=\%]
function System.advance(t) {
  // for i { for j {
  this.bodies[i]%\textsuperscript{\MakeMark{2}{3}}%.x +=
    this.bodies[j]%\textsuperscript{\MakeMark{2}{3}}%.x%\textsuperscript{\MakeMark{1}{2}}% * t
  // } }
}
\end{lstlisting}
\end{tabular}
\end{center}

\begin{itemize}
	\item[] \MakeMark{1}{2}~--- удаление проверки на \texttt{double}
	\item[] \MakeMark{2}{3}~--- замена проверки на \texttt{Body} на \texttt{null}
\end{itemize}
\end{frame}

\section{Выводы}

\ifthenelse{\equal{\presMode}{full}}{
\begin{frame}
\frametitle{Что дальше?}
\begin{center}
\begin{tabular}{p{0.5\textwidth}p{0.3\textwidth}}
Элемент&Требования\\
\hline
Обработка исключений в скомпилированном коде&специальный стековый кадр\\
\hline
Перегрузка по типам аргументов&\\
\hline
irtoc интерпретатор&\\
\hline
Конвертация на статической стороне&\texttt{mov.dyn}\\
\hline
JIT на статической стороне&\texttt{invokedynamic}\\
\hline
Конвертация замыканий&\texttt{DynamicProxy}\\
\hline
Встраивание функций&выделение безопасного общего подмножества
\end{tabular}
\end{center}
\end{frame}
	} {
}

\begin{frame}
\frametitle{Выводы}
\begin{itemize}
	\item Исследованы существующие решения
	\item Выделены затрагиваемые компоненты платформы
	\item Спроектирована архитерктура межъязыкового взаимодействия
	\item Реализован и протестирован прототип ({\color{ITMOgreen}+7461} {\color{ITMOred}-601})
	\item Получен прирост производительности в 2 раза на общепринятом тесте
	\ifthenelse{\equal{\presMode}{full}}{\item Определено направление дальнейшего развития}{}
\end{itemize}
\end{frame}

\ifthenelse{\equal{\presMode}{full}}{
\begin{frame}
\frametitle{Выявленные проблемные места платформы}
\begin{itemize}
	\item \texttt{ReferenceProcessor}
	\item Завязанность компилятора на \texttt{EntityId}
	\item Полное отделение компилятора от среды исполнения
	\item Отсутствие AnyConst
	\item Неявность мест использования предположения об однопоточности ecmascript'а
\end{itemize}
\end{frame}
}{}

\BackupBegin

\begin{frame}[plain]
\itmopolygons{
	\vfill
	\Huge{Спасибо за внимание}
	\vfill
	%\includegraphics[scale=.5]{itmo/slogan.pdf}
}
\end{frame}

\section{Дополнительные слайды}
\let\oldSection\section
\let\oldSubSection\subsection
\def\section#1{\oldSubSection{#1}}
\def\subsection#1{}

\begin{frame}
\frametitle{Дополнительный стековый кадр}
\begin{center}
\begin{tabular}{|c|@{}cr@{}|}
\hline
\multicolumn{2}{|c}{Адрес возврата}&\tikz\node[na](frameUpper){};\\
\hline
\multicolumn{2}{|c}{Предыдущий кадр $\ast$}&\\
\hline
\multicolumn{2}{|c}{Слот метаданных}&\\
\hline
\multicolumn{2}{|c}{\dots Локальные переменные}&\\
\hline\hline
\textbf{Дополнительный кадр}&\multicolumn{2}{@{}c@{}|}{\begin{tabular}{c}
Адрес возврата ($\bigstar$)\\
\hline
Предыдущий кадр $\ast$\tikz\node[na](frameLink){};\\
\hline
Специальное число\\
\hline
Cостояние потока $\ast$\\
\end{tabular}}\\
\hline\hline
\multicolumn{2}{|c}{\dots Стековые аргументы}&\\
\hline
\end{tabular}
\end{center}
($\bigstar$)~--- заполняется при каждом вызове

\begin{tikzpicture}[remember picture,overlay,cyan,rounded corners,>=stealth,shorten > =1pt,shorten <=1pt,thick]
	\draw[->, color=black] (frameLink.east) to [bend right=45] (frameUpper.east);
\end{tikzpicture}

\end{frame}

\SlidesAtEnd

\section{Пример вызова метода}

\begin{comment}

\begin{frame}
\frametitle{На статически типизированной стороне}
\begin{itemize}
	\item[\xmark] Встраиваемые кэши
	\item[\cmark] ``MethodHandle''
\end{itemize}
\end{frame}

\subsection{На динамически типизированной стороне}

\MoveSlideToEnd{
\subsection{Перегрузка}
\begin{frame}
\frametitle{Виды перегрузки}
\begin{itemize}
	\item \only<1>{по атрибутам \texttt{this}}\only<2->{\sout{по атрибутам \texttt{this}}}
	\pause
	\item по числу аргументов (elixir, erlang)
	\item по типам аргументов (typescript?)
\end{itemize}
\end{frame}

\begin{frame}
\frametitle{Перегрузка методов в GraalVM}
	Что вызовет \texttt{foo(1)}?\\
	\begin{enumerate}
		\item \lstinline|foo(arg: double): void|
		\item \lstinline|foo(arg: int): void|
		\item \lstinline|foo(arg: short): void|
	\end{enumerate}
	\pause
	Ответ:
	\begin{itemize}
		\item Из статически типизированного языка --- 2
		\item Из динамически типизированного языка --- 3
	\end{itemize}
\end{frame}

\begin{frame}
\frametitle{Решение}
\begin{itemize}
	\item Наличие перегрузки по числу аргументов
	\item Выбор набора наиболее общих сигнатур
	\begin{itemize}
		\item Главенство вызывающего языка
		\item Производительность
	\end{itemize}
\end{itemize}
\end{frame}
}
\end{comment}

\begin{frame}[fragile]
\frametitle{Данные}

\begin{center}
\begin{tabular}{|l|l|}
\hline
Статический & Динамический \\
\hline
\begin{lstlisting}
class Klass {
  func(a: int): void
  func(): void
}
\end{lstlisting} &
\begin{lstlisting}
let ins /* : Klass */
let arg /* : number */
\end{lstlisting} \\
\hline
\multicolumn{2}{|c|}{\lstinline|ins.func(arg)|}\\
\hline
\begin{lstlisting}
Klass.func.virtual(
  ins,
  arg
)
\end{lstlisting} & \only<2>{\cellcolor{lime}}\begin{lstlisting}
ins['func'](
  this=ins,
  args=(arg,)
)
\end{lstlisting} \\
\hline
\end{tabular}
\end{center}
\end{frame}

\begin{frame}
\frametitle{Промежуточное представление}
\begin{center}
\begin{tabular}{|c|c|c|c|}
\hline
№ & инструкция & ввод & пользователи \\
\hline
0 & Register Ins & & 2 \\
1 & Register Arg & & 0 \\
\rowcolor<1->{lime}2 & LoadByName ``func'' & 0 & 3 \\
\rowcolor<1->{lightgray}3 & CallDynThis & 2, 0, 1 &\\
\hline
\end{tabular}
\end{center}
\end{frame}

\begin{frame}
\frametitle{Оптимизация}
\begin{center}
\begin{tabular}{|c|c|c|r|}
\hline
№ & инструкция & ввод & пользователи \\
\hline
0 & Register Ins & & 2, 5 \\
1 & Register Arg & & 0 \\
\rowcolor{lime}\tikz\node[na](DEOPTKLASS){2}; & DeoptimizeIfNotInstance "Klass" & 0 & \only<3>{7} \\
\rowcolor<3>{red}\rowcolor<-2>{lime}\tikz\node[na](CONST){3}; & Constant "func\_const" & & 4\\
\rowcolor<3>{red}\rowcolor<-2>{lightgray}\tikz\node[na](DEOPT){4}; & DeoptimizeIfNE "func\_const" & 3 &\\
\rowcolor<3>{red}\rowcolor<-2>{lightgray}\tikz\node[na](DEOPTKLASS2){5}; & DeoptimizeIfNotInstance "Klass" & 0 & \only<-2>{7} \\
\rowcolor{lightgray}6 & DeoptimizeIfNotNumber & 1 & 7 \\
\rowcolor{lightgray}7 & CallVirtualF "Klass.func/1" & \only<-2>{5}\only<3>{2}, 6 &\\
\hline
\end{tabular}
\end{center}
\only<2>{
\begin{tikzpicture}[remember picture,overlay,cyan,rounded corners,>=stealth,shorten > =1pt,shorten <=1pt,thick]
	\draw[->, color=black] (CONST.west) to [bend right=45] (DEOPT.west);
	\draw[->, color=black] (DEOPTKLASS.west) to [bend right=60] (DEOPTKLASS2.west);
\end{tikzpicture}
}
\end{frame}


\let\section\oldSection
\let\subsection\oldSubSection
\BackupEnd

\end{document}
