\documentclass[table]{beamer}

\usepackage{fontspec}
\defaultfontfeatures{Ligatures={TeX},Renderer=Basic}
\setmainfont[Ligatures={TeX,Historic}]{Times New Roman}
\setsansfont{Arial}
\setmonofont{Courier New}

\usepackage[english,russian]{babel}
\usepackage{svg}
\usepackage{pifont}
\usepackage{tikz}

\title{Поддержка вызова методов и передачи объектов между языками со статическими и динамическими типизациями на платформе Ark}
\author{Прокопенко К.Д.}
%\institute{}
\date{2023}

\newcommand{\cmark}{\ding{51}}
\newcommand{\xmark}{\ding{55}}

\newcommand\tikznode[3][]%
{\tikz[remember picture,baseline=(#2.base)]%
	\node[minimum size=0pt,inner sep=0pt,#1](#2){#3};%
}%

\makeatletter
\def\rowcolor{{\ifnum0=`}\fi\bmr@rowcolor}
\newcommand<>{\bmr@rowcolor}{%
	\alt#1%
		{\global\let\CT@do@color\CT@@do@color\@ifnextchar[\CT@rowa\CT@rowb}%
		{\ifnum0=`{\fi}\@gooble@rowcolor}%
}
\newcommand{\@gooble@rowcolor}[2][]{\@gooble@rowcolor@}
	\newcommand{\@gooble@rowcolor@}[1][]{\@gooble@rowcolor@@}
	\newcommand{\@gooble@rowcolor@@}[1][]{\ignorespaces}
\makeatother

\begin{document}

\addtobeamertemplate{navigation symbols}{}{%
	\usebeamerfont{footline}%
	\usebeamercolor[fg]{footline}%
	\hspace{1em}%
	\insertframenumber/\inserttotalframenumber
}

\AtBeginSection[]{
	\begin{frame}
	\vfill
	\centering
	\begin{beamercolorbox}[sep=8pt,center,shadow=true,rounded=true]{title}
		\usebeamerfont{title}\insertsectionhead\par%
	\end{beamercolorbox}
	\vfill
	\end{frame}
}

\frame{\titlepage}

\begin{frame}{Outline}
	\tableofcontents
\end{frame}

\section{Обзор}

\begin{frame}
\frametitle{Актуальность}
\begin{itemize}
	\item Переиспользование
	\item Гибкость решений
	\item Кооперация
\end{itemize}
\end{frame}

\begin{frame}
\frametitle{Существующие решения}
\begin{itemize}
	\item Нативный-нативный
		\begin{itemize}
			\item{C++, Zig, Carbon}
		\end{itemize}
	\item Управляемый-нативный
		\begin{itemize}
			\item{FFI (Haskell, python, \dots)}
			\item{JNI, Project Panama}
			\item{LuaJIT}
			\item{CPPYY}
		\end{itemize}
	\item Управляемый-управляемый
		\begin{itemize}
			\item{JVM/.NET}
			\begin{itemize}
				\item{Nashorn}
				\item{Graal VM}
			\end{itemize}
		\end{itemize}
\end{itemize}
\end{frame}

\section{Представление объектов}

\subsection{NaN упаковка}

\begin{frame}
\frametitle{Общая идея}
\begin{center}
\begin{tabular}{|c|c|c|c|c|c|}
\hline
число бит & 1 & 11 & \multicolumn{3}{c|}{52} \\
\hline
значение & знак & экспонента & \multicolumn{3}{c|}{мантисса} \\
\hline
NaN & знак & 1\dots 1 & \multicolumn{3}{c|}{мантисса} \\
\hline
qNaN indefinite & знак & 1\dots 1 & 1 & \multicolumn{2}{c|}{0\dots0} \\
\hline
кодирование & знак & 1\dots 1 & 1 & тэг & значение \\
\hline
\end{tabular}
\end{center}
\end{frame}

\begin{frame}
\frametitle{реализация в Ark}
\begin{center}
\begin{tabular}{|c|c|c|}
\hline
число бит & 16 & 48 \\
\hline
значение & ``тэг'' & нагрузка \\
\hline
object & \texttt{0000} & указатель \\
\hline
float64 & \begin{tabular}{c} \texttt{0001} \\ ---- \\ \texttt{fffe} \end{tabular} & \dots \\
\hline
int32 & \texttt{ffff} & int32 \\
\hline
\end{tabular}
\end{center}
\end{frame}

\begin{frame}
\frametitle{расширение в Ark}
\begin{center}
\begin{tabular}{|c|c|c|c|}
\hline
число бит & 16 & \multicolumn{2}{c|}{48} \\
\hline
значение & \multicolumn{2}{c}{``тэг''} & нагрузка \\
\hline
object & \texttt{0000} & \texttt{0} & указатель \\
\hline
foreign object & \texttt{0000} & \texttt{1} & указатель \\
\hline
float64 & \begin{tabular}{c} \texttt{0001} \\ ---- \\ \texttt{fffe} \end{tabular} & \multicolumn{2}{c|}{\dots} \\
\hline
int32 & \texttt{ffff} & \multicolumn{2}{c|}{int32} \\
\hline
\end{tabular}
\end{center}
\end{frame}

\begin{frame}
\frametitle{Как представить статические поля и методы?}
\pause
Полиморфизм по типу объекта: если он является объектом класса, то осуществлять доступ к статическим сущностям.
\end{frame}

\subsection{На статически типизированной стороне}

\begin{frame}
	\frametitle{Новая иерархия классов}
	\begin{center}
		\includesvg{build/res/dot/statically-typed-tree.dot}
	\end{center}
\end{frame}

\section{Управление потоками}

\begin{frame}
\frametitle{Состояние потока}
\begin{itemize}
	\item Инкапсулирует данные необходимые для исполнения
	\item Хранится в переменной локльной для потока или в регистре
	\item Добавляется новый переход
	\pause
	\begin{itemize}
		\item необходимо переключать в мостах
		\item отсутствие гонок данных
	\end{itemize}
\end{itemize}
\end{frame}


\section{Вызов методов}

\begin{frame}
\frametitle{На статически типизированной стороне}
\begin{itemize}
	\item[\xmark] Встраиваемые кэши
	\item[\cmark] ``MethodHandle''
\end{itemize}
\end{frame}

\begin{frame}
\frametitle{На динамически типизированной стороне}
\begin{itemize}
	\item[] \tikznode{V1}var.fu\tikznode{F1}nc(a\tikznode{A1}rg)
	\item[]
	\item[] (var.fu\tikznode{F2}nc)(this=v\tikznode{V2}ar, a\tikznode{A2}rg)
\end{itemize}
\begin{tikzpicture}[remember picture,overlay,cyan,rounded corners,>=stealth,shorten > =1pt,shorten <=1pt,thick]
	\draw[->] (A1.south) -- (A2.north);
	\draw[->] (F1.south) -- (F2.north);
	\draw[->] (V1.south) -- (V2.north);
\end{tikzpicture}
\end{frame}

\begin{frame}
\frametitle{Упрощенные преобразования}
\begin{tabular}{|c|c|c|c|}
№ & инструкция & ввод & пользователи \\
0 & Register Var & & 2 \\
1 & Register Arg & & 0 \\
\rowcolor<1->{lime}2 & LoadByName "func" & 0 & 3 \\
\rowcolor<1->{lightgray}3 & CallDynThis & 2, 0, 1 &
\end{tabular}
\end{frame}

\begin{frame}
\begin{tabular}{|c|c|c|c|}
№ & инструкция & ввод & пользователи \\
0 & Register Var & & 2, 5 \\
1 & Register Arg & & 0 \\
\rowcolor{lime}2 & DeoptimizeIfNotInstance "Klass" & 0 & \only<2>{7} \\
\rowcolor<2>{red}\rowcolor<1>{lime}3 & Constant "func\_const" & & \\
\rowcolor<2>{red}\rowcolor<1>{lightgray}4 & DeoptimizeIfNE "func\_const" & 3 & \\
\rowcolor<2>{red}\rowcolor<1>{lightgray}5 & DeoptimizeIfNotInstance "Klass" & 0 & \only<1>{7} \\
\rowcolor{lightgray}6 & DeoptimizeIfNotNumber & 1 & 7 \\
\rowcolor{lightgray}7 & CallVirtual "Klass.func" & \only<1>{5}\only<2>{2}, 6 &
\end{tabular}
\end{frame}

\begin{frame}
\frametitle{Перегрузка методов в динамически типизированных языках}
\begin{itemize}
	\item по числу аргументов (elixir, erlang)
	\item по типам аргументов (typescript?)
\end{itemize}
\end{frame}

\begin{frame}
\frametitle{Перегрузка методов в GraalVM}
	Что вызовет \texttt{foo(1)} вызванное из динамически типизированного языка?\\
	\begin{enumerate}
		\item \texttt{void foo(double);}
		\item \texttt{void foo(int);}
		\item \texttt{void foo(char);}
	\end{enumerate}
	\pause
	Ответ: 3
\end{frame}

\section{Управление памятью}

\begin{frame}
\frametitle{Обход стека}
\begin{itemize}
	\item определение типа метода
	\item выбор стеков из состояний потока
\end{itemize}
\end{frame}

\begin{frame}
\frametitle{Иерархия маркеров --- старая}
\begin{center}
\includesvg{build/res/dot/marker_was.dot}
\end{center}
\end{frame}
\begin{frame}
\frametitle{Иерархия маркеров --- новая}
\begin{center}
\scalebox{0.6}{
\includesvg{build/res/dot/marker_new.dot}
}
\end{center}
\end{frame}

\begin{frame}
	\vfill
	\centering
	\begin{beamercolorbox}[sep=8pt,center,shadow=true,rounded=true]{title}
		\usebeamerfont{title}Спасибо за внимание
		\only<2>{\\ Вопросы?}
	\end{beamercolorbox}
	\vfill
\end{frame}

\end{document}
